
\begin{center}
\footnotesize
\begin{tikzpicture}[
  ->,
  double,
  very thick]

\node (test) at (0,0) [class, text width=\W{27}]{\textbf{IntermNormWaveFunction}
\nodepart{second}
+ norm : float\\
+ singles : list[np.array]\\
+ doubles : list[np.array]
\nodepart{third}
+ get\_irrep ( i : int, alpha\_orb : int )\\
+ ij\_from\_N ( N : int )\\
+ N\_from\_ij ( i : int, j : int, i\_irrep : int, j\_irrep : int, exc\_type : str)};

\node (NormCI_WF) at (0,-5) [class, text width=\W{27}]{\textbf{NormCI\_WaveFunction}
\nodepart{second}
+ has\_FCI\_structure : bool};

\node (CISD_WF) at (0,-8) [class, text width=\W{27}]{\textbf{CISD\_WaveFunction}
\nodepart{second}
+ C0 : float\\
+ Cs ; list of 2D np.ndarrays\\
+ Cd ; list of 2D np.ndarrays\\
+ Csd ; list of 4D np.ndarrays};

\node (FCI_WF) at (0,-12) [class, text width=\W{27}]{\textbf{WaveFunctionFCI}
\nodepart{second}
};

\node (SlaterD) at (\X{40},-10) [class, text width=\W{27}]{\textbf{SlaterDet}
\nodepart{second}
+ c : float\\
+ occupation : list of np.arrays of int};

\node (ExcSlaterD) at (\X{40},-13) [class, text width=\W{27}]{\textbf{ExcitedSlaterDet}
\nodepart{second}
 ??};



\node (test2) at (\X{40},\Y{0}) [class, text width=\W{27}]{\textit{\textbf{WaveFunction}}
\nodepart{second}
+ restricted : bool\\
+ point\_group : str\\
+ n\_irre : int\\
+ n\_core : OrbitalsSets\\
+ n\_act : OrbitalsSets\\
+ orb\_dim : OrbitalsSets\\
+ ref\_occ : OrbitalsSets\\
+ n\_ext : OrbitalsSets\\
+ n\_corr\_orb : OrbitalsSets\\
+ n\_alpha, n\_beta, n\_elec, n\_corr\_alpha, n\_corr\_beta, n\_corr\_elec, n\_orb, n\_orb\_nocore: int\\
+ WF\_type : str\\
+ source : str};

\node (test3) at (\X{80},\Y{0}) [class, text width=\W{15}]{\textbf{OrbitalSets}
\nodepart{second}
+ occ\_type : str
};

\node (MO) at (\X{0},\Y{85}) [class, text width=\W{20}]{\textbf{MolecularOrbital}
\nodepart{second}
+ name : str\\
+ n\_irrep : int\\
+ restricted : bool\\
+ in\_the\_basis : str\\
+ sym\_adapted\_basis : bool\\
+ energies : 1D ndarray
};

\node (Atom) at (\X{50},\Y{35}) [class, text width=\W{20}]{\textbf{Atom}
\nodepart{second}
+ element : str\\
+ coord : array of floats\\
+ label : str
};


\node (MG) at (\X{80},\Y{30}) [class, text width=\W{20}]{\textbf{MolecularGeometry}
\nodepart{second}
+ name : str\\
- \_atoms : array of Atoms\\
+ atomic\_basis\_set : str\\
+ integrals : Integrals\\
+ charge : int = 0\\
- \_nuc\_rep : float\\
- \_n\_elec : int
};

\node (Int) at (\X{80},\Y{80}) [class, text width=\W{20}]{\textbf{Integrals}
\nodepart{second}
+ basis\_set : str\\
+ mol\_geo : MolecularOrbital\\
+ n\_func : int\\
+ S : np.ndarray (??)\\
+ h : np.ndarray (??)\\
+ g : np.ndarray (??)\\
+ X : (??)
};

\node (Note) at (\X{65},\Y{55}) [shape=rectangle, draw, text width=\W{25}]{Note: MolecularGeomtry does not need to know the basis used\\
  and more than one set of integrals can be generated for the same Geom.\\
This link can be unidiretional};

\node (Note2) at (\X{25},\Y{85}) [shape=rectangle, draw, text width=\W{25}]{Note: One or more MolecularOrbital can be linked with an Integrals object,\\
  but Integrals doesn't need to be linked to any of them.};



\draw [-{Triangle[open]}] (test) -- (test2);
\draw [-{Triangle[open]}] (NormCI_WF) -- (test2);
\draw [-{Triangle[open]}] (CISD_WF) -- (test2);
\draw [-{Triangle[open]}] (FCI_WF) -- (test2);
%__
\draw [<-{Turned Square[open]}] (test3) -- (test2);
\node [below left = -0.6cm and 0.1cm of test3] {6};
\node [above left = -0.6cm and 0.1cm of test3] {define orbitals};
\node [below right = -5cm and 0.1cm of test2] {1};
%--
%__
\draw [{Turned Square[open]}->] (MG) -- (Atom);
\node [above left = -3.1cm and 0.1cm of MG] {1};
\node [below right = -2.6cm and 0.05cm of Atom] {formed by};
\node [above right = -1.9cm and 0.1cm of Atom] {0..};
%--
%__
\draw [-] (Int)  -- (MG);
\node [below right = 0.1cm and -2cm of MG]{1};
\node [above left = 0.1cm and -2cm of Int]{1};
%--



\end{tikzpicture}
\end{center}

%%% Local Variables:
%%% mode: latex
%%% TeX-master: "grassmann_doc.tex"
%%% End:
