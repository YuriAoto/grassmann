\hypertarget{chap:input}{}
\chapter{Input}
\label{sec:input}
\chaptermark{Input}
\chapterauthor{}


\section{General considerations}

All input information are passed in one of two forms:
\begin{itemize}
\item \emph{key}/\emph{value}:
  The \emph{value} is used for that \emph{key},
  e.g. \emph{key}=\verb+method+ and \emph{value}=\verb+Hartree_Fock+ sets Hartree-Fock as the method.
\item \emph{flag}:
  Activates \emph{flag},
  e.g. \emph{flag}=\verb+restricted+ sets a restricted type of calculation.
\end{itemize}
A list of all possible arguments are in \ref{sec:inputargs}.

These input information can be passed to Grassmann through arguments in the command line arguments
and/or read from input file(s). Every argument can be given in either of the two forms:
\begin{itemize}
\item In the command line, keys and flags should be preceded by \verb+--+.
  If of type key/value, the value comes after the key, separated by a space or a equal (\verb+=+) sign.
\item Instead of using the command line to pass (perhaps several) arguments, this can be made
  in an input file, whose name(s) can be passed in the command line.
  In this case, the arguments should be one in each line, with an equal sign (\verb+=+) separating
  key/values pairs.
  The character ``\verb+#+'' is used to begin a comment, and the remaining of the line is ignored.
\end{itemize}
Both ways can be used in conjunction, with several and nested input files.
If the same argument is given more than once the precedence is ``use what is given last''.

\subsection{Examples}

Sets \verb+Hartree_Fock+ as method and \verb+H2.xyz+ as geometry (file), only from the command line:
\begin{lstlisting}[style=shstyint]
@\shPr{}@ @\grassmannexec{}@ --method Hartree_Fock --geometry H2.xyz
\end{lstlisting}

Sets \verb+Hartree_Fock+ as method and \verb+H2.xyz+ as geometry (file), from a input file:
\begin{lstlisting}[style=shstyint]
@\shPr{}@ cat myfile.inp
method =Hartree_Fock # method
geometry = H2.xyz
@\shPr{}@ @\grassmannexec{}@ myfile.inp
\end{lstlisting}

Sets \verb+Hartree_Fock+ as method and \verb+H2.xyz+ as geometry (file), from both command line and input file:
\begin{lstlisting}[style=shstyint]
@\shPr{}@ cat myfile.inp
method =Hartree_Fock # method
@\shPr{}@ @\grassmannexec{}@ myfile.inp --geometry H2.xyz
\end{lstlisting}

Sets \verb+Hartree_Fock+ as method and \verb+H2.xyz+ as geometry (file),
from both command line and input file.
Note that since \verb+H2.xyz+ is passed after the file, that contains \verb+whatever+,
for the key geometry, this is what Grassmann uses:
\begin{lstlisting}[style=shstyint]
@\shPr{}@ cat myfile.inp
method =Hartree_Fock # method
geometry = whatever
@\shPr{}@ @\grassmannexec{}@ myfile.inp --geometry H2.xyz
\end{lstlisting}

Error (assuming that there is no file named ``\verb+whatever+''),
because \verb+H2.xyz+ is passed before the file, that contains \verb+whatever+ for the key geometry:
\begin{lstlisting}[style=shstyint]
@\shPr{}@ cat myfile.inp
method =Hartree_Fock # method
geometry = whatever
@\shPr{}@ @\grassmannexec{}@ --geometry H2.xyz myfile.inp
\end{lstlisting}


\section{Geometry files}
\label{sec:geometry}
Geometry can be specified in standard xyz file:
The first line contains the number of atoms;
The second is a title/comment, that is ignored;
From the third line a list of all atoms,
with the element symbol and its three Cartesian coordinates,in \AA, separated with spaces.
Optionally, a number can be put together with the element symbol to differentiate
among atoms of the same element.

Example:
\begin{lstlisting}[style=filestyint]
2
The hydrogen molecule
H1   0.0     0.0    0.0
H2   0.74    0.0    0.0
\end{lstlisting}


\section{Input arguments}
\label{sec:inputargs}

In the following, all arguments of Grassmann are listed and briefly described.

\subsection{key/value arguments}

\begin{itemize}
  % ===================================
\item \verb+method+

  \emph{Possible values}: \verb+dist_Grassmann+ and \verb+Hartree_Fock+

  \emph{Default value}: \verb+dist_Grassmann+

  The main calculation.

  % ===================================
\item \verb+geometry+

  \emph{Possible values}: A file name.

  \emph{Default value}: None.

  The file with the geometry description, see Sect.~\ref{sec:geometry}.

  % ===================================
\item \verb+molpro_output+

  \emph{Possible values}: A file name.

  \emph{Default value}: None.

  The Molpro file name with the wave function to be used in the calculation.

  % ===================================
\item \verb+memory+

  \emph{Possible values}: a number, optionally followed by a memory unit
  (\verb+B+, \verb+kB+, \verb+MB+, \verb+GB+, or \verb+TB+).

  \emph{Default value}: \verb+100.0kB+.

  The maximum memory that will be permitted to be allocated.
  If only a number is passed, it is assumed to be in kB.
  
  % ===================================
\item \verb+basis+

  \emph{Possible values}: Any basis that is in the directory \verb+bases+ of IR-WMME or any basis that van be obtained from \href{https://www.basissetexchange.org/}{https://www.basissetexchange.org/}.

  \emph{Default value}: cc-pVDZ.

  The basis set to be used in the calculation.


  % ===================================
\item \verb+ini_orb+

  \emph{Possible values}: Molpro's ``put'' xml file or npz file.

  \emph{Default value}: None.
  
  Initial orbital to be used in orbital optimisations.
  
  % ===================================
\item \verb+HF_orb+

  \emph{Possible values}: Molpro's ``put'' xml file.

  \emph{Default value}: None.

  The Hartree-Fock orbitals

  % ===================================
\item \verb+WF_orb+

  \emph{Possible values}: Molpro's ``put'' xml file.

  \emph{Default value}: None.
  
  The orbitals used to construct the wave function (that is read from \verb+molpro_output+)
  
  % ===================================
\item \verb+WF_templ+

  \emph{Possible values}: Molpro output.

  \emph{Default value}: None.
  
  A file with a FCI wave function, that will be used as ``template''.
  
  % ===================================
\item \verb+maxiter+

  \emph{Possible values}: Integer.

  \emph{Default value}: Method dependent.

  Maximum number of iterations

  % ===================================
\item \verb+algorithm+

  \emph{Possible values}: \verb+orb_rotations+, \verb+general_Absil+, \verb+CISD_Absil+

  \emph{Default value}: \verb+CISD_Absil+.

  The algorithm, to be used in a min dist (at the Grassmannian) calculation.

  % ===================================
\item \verb+state+

  \emph{Possible values}: An electronic state in Molpro notation.

  \emph{Default value}: None.

  The state to be read from \verb+molpro_output+. If None, uses the first.

  % ===================================
\item \verb+loglevel+

  \emph{Possible values}: Integer or a log level name.

  \emph{Default value}: \verb+warning+.
  
  The log level, see Sect~\ref{sec:controllog}.
  
  % ===================================
\item \verb+logfilter+

  \emph{Possible values}: A regular expression.

  \emph{Default value}: None.
  
  A regular expression filter for function names that will be logged,
  see Sect~\ref{sec:controllog}.
  
  % ===================================
\item \verb+out_extension+

  \emph{Possible values}: String.

  \emph{Default value}: \verb+.gr+.
  
  The extension to be used for the output file. See Sect~\ref{sec:outname}
  
  % ===================================
\item \verb+log_extension+

  \emph{Possible values}: String.

  \emph{Default value}: \verb+.grlog+.
  
  The extension to be used for the log file. See Sect~\ref{sec:outname}
  
  % ===================================
\item \verb+dir_extension+

  \emph{Possible values}: String.

  \emph{Default value}: \verb+.grdir+.
  
  The extension to be used for the output directory. See Sect~\ref{sec:outname}

  % ===================================
\item \verb+output+

  \emph{Possible values}: String.

  \emph{Default value}: None.

  The output file name. See Sect~\ref{sec:outname}

  
\end{itemize}

\subsection{flag arguments}

\begin{itemize}
  % ===================================
\item \verb+restricted+

  \emph{Default value}: Depends on the method.

  Do a spin restricted calculation.

  % ===================================
\item \verb+save_final_orb+

  \emph{Default value}: False.

  If set, saves the final orbital.

  % ===================================
\item \verb+save_full_orb+

  \emph{Default value}: False.

  If set, saves the virtual orbitals too.

  % ===================================
\item \verb+save_all_orb+

  \emph{Default value}: False.

  If set, saves the orbitals of all iterations.

\end{itemize}


\section{Environment variables}
\label{sec:envvar}

Grassmann is affected by the following environment variables:
\begin{itemize}
\item \verb+GR_MEMORY+: Sets the maximum memory, in kB;
\item \verb+GR_TESTS_CATEG+: The tests categories to run (see \ref{});
\item \verb+GR_IR_WMME_DIR+: The directory of the program ir-wmme (see Sect.~\ref{sec:irwmme}).
\end{itemize}


%%% Local Variables:
%%% mode: latex
%%% TeX-master: "grassmann_doc.tex"
%%% End:


