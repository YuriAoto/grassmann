\hypertarget{chap:ccsd}{}
\chapter{Closed shell coupled cluster with singles and doubles}
\label{sec:ccsd}
\chaptermark{CCSD}
\chapterauthor{}

{History:
  
  \begin{tabular}{l@{ - }l}
     2020 & Start\\
  \end{tabular}
}\vspace{3cm}


\section{Notation}

\begin{center}
  \begin{tabular}{ll}
    \hline
    \Hamilt    & The electronic Hamiltonian \\
    $g_{pqrs}$ & Two electron integrals over spatial orbitals \\
               & $= \int dr_1 dr_2 \frac{\phi^*_p(r_1)\phi^*_r(r_2)\phi_q(r_1)\phi_s(r_2)}{r_{12}}$\\
    $L_{pqrs}$ & $= 2g_{pqrs} - g_{psrq}$\\
               & (as suggested in \cite{})\\
    \hline
  \end{tabular}
\end{center}

\newpage
\section{Introduction}

The general equations of coupled cluster theory are ``deceptively simple'',
as R. Bartlet has stated \cite{}:
\begin{equation}
  E = \mel{\Phi_0}{e^{-T}\Hamilt e^T}{\Phi_0}
\end{equation}
\begin{equation}\label{eq:gen_ccsd_ampl_eq}
  0 = \mel{\Phi_\mu}{e^{-T}\Hamilt e^T}{\Phi_0}\,\\\
\end{equation}
where $\Phi_0$ is the reference Slater determinant,
$\Hamilt$ is the electronic Hamiltonian,
and
\begin{equation}
  T = \sum_\mu t_\mu \tau_\mu
\end{equation}
is the cluster operator, with amplitudes $t_\mu$ associated to the excitations $\tau_\mu$.
The amplitudes are obtained after solving equations \eqref{eq:gen_ccsd_ampl_eq},
where $\ket{\Phi_\mu} = \tau_\mu \ket{\Phi_0}$ is the Slater determinant
obtained after acting the excitation $\tau_\mu$ on top of $\ket{\Phi_0}$.

These equations can be made specific to a particular case,
and put in terms of the molecular integrals.
This is done in Section 13.7 of reference \cite{}, that we followed closely in the present
implementation.
Below we show the main equations.
Some of them are directly obtained from \cite{},
whereas in other cases we have adapted them to a form that is more directly related to our implementation.
In section \ref{sec:ccsd_flowchart} we have a flowchart that show,
as precisely as we could make,
the connection between the equations and the code.


\section{Equations}

\subsection{Energy}
\hypertarget{sec:ccsd_energy}{}
\label{sec:ccsd_energy}

\begin{equation}
  \begin{split}
    E =& \mel{\Phi_0}{e^{-T}\Hamilt e^T}{\Phi_0}\\
    =& \mel{\Phi_0}{\Hamilt e^T}{\Phi_0}\\
    =& E_{HF} + \sum_{aibj}(t_{ij}^{ab} + t_i^at_j^b) L_{iajb}
  \end{split}
\end{equation}


\subsection{Residuals}
\hypertarget{sec:ccsd_res}{}
\label{sec:ccsd_res}

\subsubsection{Singles}
\hypertarget{sec:ccsd_res_sing}{}
\label{sec:ccsd_res_sing}

\begin{equation}
  \begin{split}
    \Omega_i^a = ...
  \end{split}
\end{equation}

\subsubsection{Doubles}
\hypertarget{sec:ccsd_res_doub}{}
\label{sec:ccsd_res_doub}

\begin{equation}
  \begin{split}
    \Omega_{ij}^{ab} = ...
  \end{split}
\end{equation}


\subsection{Amplitudes update}
\hypertarget{sec:ccsd_update}{}
\label{sec:ccsd_update}

\begin{equation}
  t_\mu = t_\mu + \Delta t_\mu
\end{equation}

\begin{equation}
  \Delta t_i^a = \frac{\Omega_i^a(\mathbf{t})}{\epsilon_a - \epsilon_i}
\end{equation}

\begin{equation}
  \Delta t_{ij}^{ab} = \frac{\Omega_{ij}^{ab}(\mathbf{t})}
  {\epsilon_a + \epsilon_b - \epsilon_i - \epsilon_j}
\end{equation}



\newpage
\section{Flowchart of the algorithm}
\label{sec:ccsd_flowchart}

\newcommand{\C}[1]{0.0+#1*4.0}
\newcommand{\T}[1]{0.0-#1*2.5}
\newcommand{\W}[1]{#1*3}

\newcommand{\toBlue}{\color{blue}}
\newcommand{\toBlack}{\color{black}}

\newcommand{\pyVar}[1]{\textit{$\langle$#1$\rangle$}}


\tikzset{
  hyperlink node/.style={
    alias=sourcenode,
    append after command={
      let     \p1 = (sourcenode.north west),
      \p2=(sourcenode.south east),
      \n1={\x2-\x1},
      \n2={\y1-\y2} in
      node [inner sep=0pt, outer sep=0pt,anchor=north west,at=(\p1)] {\hyperlink{#1}{\XeTeXLinkBox{\phantom{\rule{\n1}{\n2}}}}}
      % xelatex needs \XeTeXLinkBox, won't create a link unless it
      % finds text --- rules don't work without \XeTeXLinkBox.
      % Still builds correctly with pdflatex and lualatex
    }
  }
}

\begin{center}
\footnotesize
\begin{tikzpicture}[
  ->,
  double,
  very thick,
  proj/.style={
    align=center,
    font=\bfseries,
    color=red
  },
  projBox/.style={
    thin
  },
  algstep/.style={
    align=center,
    anchor=north,
    rectangle,
    minimum size=6mm,
    rounded corners=3mm,
    very thick,
    draw=black!50,
    top color=white,
    bottom color=black!20
  },
  algmultistep/.style={
    align=center,
    anchor=north,
    rectangle,
    minimum size=6mm,
    rounded corners=3mm,
    very thick,
    draw=black!50,
    top color=white,
    bottom color=white
  }
  ]

  
  \def\lwdB{0.2cm};
  \def\lwd{0.1cm};
  \def\lwdE{0.0cm}


  % ======
  \node (init)  at (\C{0.3}, \T{-1.5})
  [algstep, text width=\W{80}, hyperlink node=sec:]
  {
    \textbf{Initialisation}\\\vspace{\lwdB}
    \begin{minipage}{0.4\textwidth}
      {\footnotesize \texttt{ccsd.}}\\\vspace{\lwd}
      
    \end{minipage}
  };
  % ------

  
  % ======
  \node (ener)  at (\C{-0.0}, \T{0.0})
  [algstep, text width=\W{80}, hyperlink node=sec:ccsd_energy]
  {
    \textbf{Energy}\\\vspace{\lwdB}
    \begin{minipage}{0.4\textwidth}
      {\footnotesize \texttt{ccsd.get\_energy}}\\\vspace{\lwd}
      
    \end{minipage}
  };
  % ------

  
  % ================================================
  \node (res) at (\C{0.0}, \T{1.0})
  [algmultistep, text width=\W{150}, minimum height=4cm, hyperlink node=sec:ccsd_res]
  {
    \textbf{Set residual}\\\vspace{\lwdB}
    \begin{minipage}[t][9cm]{0.4\textwidth}
      \centering
      {\footnotesize \texttt{ccsd.equation}}\\\vspace{\lwd}
    \end{minipage}
  };


  % ======
  \node (resterm1) at (\C{-1.0}, \T{2.0})
  [algstep, text width=\W{50}, hyperlink node=sec:ccsd_res_sing]
  {
    \textbf{Set res. singles ($\Omega_i^a$)}\\\vspace{\lwdB}
    {\footnotesize \texttt{ccsd.equation.res\_singles}}\\\vspace{\lwd}
  };
  % ------

  % ======
  \node (resterm2) at (\C{1.0}, \T{2.0})
  [algstep, text width=\W{50}, hyperlink node=sec:ccsd_res_doub]
  {
    \textbf{Set res. doubles ($\Omega_{ij}^{ab}$)}\\\vspace{\lwdB}
    {\footnotesize \texttt{ccsd.equation.res\_doubles}}\\\vspace{\lwd}
  };
  % ------

  % ======
  \node (resterm3) at (\C{0.0}, \T{3.5})
  [algstep, text width=\W{30}, hyperlink node=sec:residual_eq3]
  {
    \textbf{Set term3}\\\vspace{\lwdB}
    {\footnotesize \texttt{ccsd.equation.term3}}\\\vspace{\lwd}
  };
  % ------

    \draw [->] (resterm1.270) .. controls +(down:1.5cm) and +(up:1.5cm) .. (resterm3.90);
    \draw [->] (resterm2.270) .. controls +(down:1.5cm) and +(up:1.5cm) .. (resterm3.90);
    
  
  % ------------------------------------------------

  
  % ======
  \node (update) at (\C{0.0}, \T{6.0})
  [algstep, text width=\W{90}, hyperlink node=sec:ccsd_update]
  {
    \textbf{Update amplitudes}\\\vspace{\lwdB}
    \begin{minipage}{0.4\textwidth}
      \centering
      {\footnotesize \texttt{ccsd.step}}\\\vspace{\lwd}
    \end{minipage}
  };
  % ------



  \draw [->] (init.270) -- (ener.50);

  \draw [->] (ener.270) .. controls +(down:1.5cm) and +(up:1.5cm) .. (res.90);
  \draw [->] (res.270) .. controls +(down:1.5cm) and +(up:1.5cm) .. (update.90);

  \draw [->] (update.230) -- (\C{-0.3}, \T{7.0})  --  (\C{-2.0}, \T{7.0}) --
  (\C{-2.0}, \T{-0.5}) -- (\C{-0.3}, \T{-0.5}) -- (ener.130);
  % ======
  \node at (\C{-0.5}, \T{6.8}) []
  {$|\mathbf{\Omega}| > \epsilon$};
  % ------

  \draw [->] (update.310) -- (\C{0.3}, \T{7.0})  --  (\C{1.0}, \T{7.0});
  % ======
  \node at (\C{0.5}, \T{6.8}) []
  {$|\mathbf{\Omega}| < \epsilon$};
  % ------

\end{tikzpicture}
\end{center}

 	

%%% Local Variables:
%%% mode: latex
%%% TeX-master: "grassmann_doc.tex"
%%% End:




%%% Local Variables:
%%% mode: latex
%%% TeX-master: "grassmann_doc.tex"
%%% End:
