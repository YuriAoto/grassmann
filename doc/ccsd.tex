\hypertarget{chap:ccsd}{}
\chapter{Closed shell coupled cluster with singles and doubles}
\label{sec:ccsd}
\chaptermark{CCSD}
\chapterauthor{}

{History:
  
  \begin{tabular}{l@{ - }l}
     2020 & Start\\
  \end{tabular}
}\vspace{3cm}


\section{Notation}

\begin{center}
  \begin{tabular}{ll}
    \hline
    \Hamilt    & The electronic Hamiltonian \\
    $g_{pqrs}$ & Two electron integrals over spatial orbitals \\
               & $= \int dr_1 dr_2 \frac{\phi^*_p(r_1)\phi^*_r(r_2)\phi_q(r_1)\phi_s(r_2)}{r_{12}}$\\
    $L_{pqrs}$ & $= 2g_{pqrs} - g_{psrq}$\\
               & (as suggested in \cite{})\\
    \hline
  \end{tabular}
\end{center}

\newpage
\section{Introduction}

The general equations of coupled cluster theory are ``deceptively simple'',
as R. Bartlet has stated \cite{}:
\begin{equation}
  E = \mel{\Phi_0}{e^{-T}\Hamilt e^T}{\Phi_0}
\end{equation}
\begin{equation}\label{eq:gen_ccsd_ampl_eq}
  0 = \mel{\Phi_\mu}{e^{-T}\Hamilt e^T}{\Phi_0}\,\\\
\end{equation}
where $\Phi_0$ is the reference Slater determinant,
$\Hamilt$ is the electronic Hamiltonian,
and
\begin{equation}
  T = \sum_\mu t_\mu \tau_\mu
\end{equation}
is the cluster operator, with amplitudes $t_\mu$ associated to the excitations $\tau_\mu$.
The amplitudes are obtained after solving equations \eqref{eq:gen_ccsd_ampl_eq},
where $\ket{\Phi_\mu} = \tau_\mu \ket{\Phi_0}$ is the Slater determinant
obtained after acting the excitation $\tau_\mu$ on top of $\ket{\Phi_0}$.

These equations can be made specific to a particular case,
and put in terms of the molecular integrals.
This is done in Section 13.7 of reference \cite{}, that we followed closely in the present
implementation.
Below we show the main equations.
Some of them are directly obtained from \cite{},
whereas in other cases we have adapted them to a form that is more directly related to our implementation.
In section \ref{sec:ccsd_flowchart} we have a flowchart that show,
as precisely as we could make,
the connection between the equations and the code.

\section{The wave function}

The CCD and the CCSD wave functions are:
\begin{equation}
  \Psi_{CCSD} = e^{T_1 + T_2}\Phi_0
\end{equation}
\begin{equation}
  \Psi_{CCD} = e^{T_2}\Phi_0
\end{equation}
The operators are defined as:
\begin{equation}
  T_1 = \sum_{i,a} t_i^a E_i^a = \sum_{i,a} t_i^a (a_{a\alpha}^\dagger a_{i\alpha} + a_{a\beta}^\dagger a_{i\beta})
\end{equation}
\begin{equation}
  \begin{split}
    T_2 =& \frac{1}{2}\sum_{i,j,a,b} t_{ij}^{ab} E_i^aE_j^b\\
    =& \sum_{i \le j,a,b} t_{ij}^{ab} E_i^aE_j^b\\
    =& \sum_{i \le j,a,b} t_{ij}^{ab} (a_{a\alpha}^\dagger a_{i\alpha} + a_{a\beta}^\dagger a_{i\beta})
    (a_{b\alpha}^\dagger a_{j\alpha} + a_{b\beta}^\dagger a_{j\beta})
  \end{split}
\end{equation}

The wave function is stored in the class \texttt{wave\_functions.interm\_norm.IntermNormWF}


\section{Equations}

\subsection{Intermediary Matrices}
\hypertarget{sec:ccsd_inter_matrix}{}
\label{sec:ccsd_inter_matix}
We can define three matrices that speed up calculations at the cost of memory.
They are:
\begin{align}
    u_{ij}^{ab}&=2t_{ij}^{ab}-t_{ji}^{ab}\\
    L_{pqrs}&=2g_{pqrs}-g_{psrq}\\
    F_{mn}^I&=h_{mn}+\sum_l(2g_{mnll}-g_{mlln})=h_{mn}+\sum_lL_{mnll}
\end{align}

The T1-transformed $L$ and $F^I$ matrices (${\tilde L}$ and ${\tilde F}^I$) can be obtained using the T1-transformed $h$ and $g$ elements or transforming these matrices.
The $F^I$ is transformed as the $h$ matrix, eq.~\ref{eq:h_transformation}, and $L$ as the $g$ matrix, eq.~\ref{eq:g_transformation}.
((Which is the fastest way? Probably use the $h$ and $g$ transformed elements, this way we don't need matrices multiplication.))

\subsection{Energy}
\hypertarget{sec:ccsd_energy}{}
\label{sec:ccsd_energy}

\begin{equation}
  \begin{split}
    E =& \mel{\Phi_0}{e^{-T}\Hamilt e^T}{\Phi_0}\\
    =& \mel{\Phi_0}{\Hamilt e^T}{\Phi_0}\\
    =& E_{HF} + \sum_{aibj}(t_{ij}^{ab} + t_i^at_j^b) L_{iajb}
  \end{split}
\end{equation}

Note that $L_{iajb}={\tilde L}_{iajb}$, so in the implementation both matrices can be used. 

\subsection{T1-Transformed MO Integrals}
\hypertarget{sec:ccsd_t1_trans}{}
\label{sec:ccsd_t1_trans}

The T1-transformation depends on the $\mathbf{t}_1$ matrix defined as
\begin{gather}
  \mathbf{t}_1=
  \begin{bmatrix}
    0     & 0\\
    t_i^a & 0
  \end{bmatrix},
\end{gather}
where $t_i^a$ are the amplitudes associated to the single excitation.
Be careful with the index, because $(\mathbf{t}_1)_{ai}=t_i^a$ (maybe we should change this).
The one-electron T1-transformed molecular orbital integral can be writen as
\begin{equation}
  \label{eq:h_transformation}
  {\tilde h}_{pq}=h_{pq}-\sum_r(\mathbf{t}_1)_{pr}h_{rq}+\sum_sh_{ps}(\mathbf{t}_1^T)_{qs}-\sum_{rs}(\mathbf{t}_1)_{pr}h_{rs}(\mathbf{t}_1^T)_{qs}
\end{equation}
However, there is no need to construct the $\mathbf{t}_1$ matrix if we split the last equation in four cases:
\begin{align}
  {\tilde h}_{ij}&=h_{ij}+\sum_ch_{ic}t_{j}^{c}\\
  {\tilde h}_{ai}&=h_{ai}-\sum_kt_{k}^{a}h_{ki}+\sum_ch_{ac}t_{i}^{c}-\sum_{ck}t_{k}^{a}h_{kc}t_{i}^{c}\\
  {\tilde h}_{ia}&=h_{ia}\\
  {\tilde h}_{ab}&=h_{ab}-\sum_kt_{k}^{a}h_{kb}
\end{align}

For the two-electron molecular orbital integral we have a similar expression:
\begin{equation}
    \label{eq:g_transformation}
  \begin{split}
    {\tilde g}_{pqrs}=g_{pqrs}&-\sum_t(\mathbf{t}_1)_{pt}g_{tqrs}+\sum_ug_{purs}(\mathbf{t}_1^T)_{qu}-\sum_m(\mathbf{t}_1)_{rm}g_{pqms}+\sum_ng_{pqrn}(\mathbf{t}_1^T)_{sn}\\
    &-\sum_{tu}(\mathbf{t}_1)_{pt}g_{turs}(\mathbf{t}_1^T)_{qu}+\sum_{tm}(\mathbf{t}_1)_{pt}(\mathbf{t}_1)_{rm}g_{tqms}-\sum_{tn}(\mathbf{t}_1)_{pt}g_{tqrn}(\mathbf{t}_1^T)_{sn}\\
    &-\sum_{um}(\mathbf{t}_1)_{rm}g_{tums}(\mathbf{t}_1^T)_{qu}+\sum_{un}g_{purn}(\mathbf{t}_1^T)_{qu}(\mathbf{t}_1^T)_{sn}-\sum_{mn}(\mathbf{t}_1)_{rm}g_{pq}(\mathbf{t}_1^T)_{sn}\\
    &+\sum_{tum}(\mathbf{t}_1)_{pt}(\mathbf{t}_1)_{rm}g_{tums}(\mathbf{t}_1^T)_{qu}-\sum_{tun}(\mathbf{t}_1)_{pt}g_{turn}(\mathbf{t}_1^T)_{qu}(\mathbf{t}_1^T)_{sn}\\
    &+\sum_{tmn}(\mathbf{t}_1)_{pt}(\mathbf{t}_1)_{rm}g_{tqmn}(\mathbf{t}_1^T)_{sn}-\sum_{umn}(\mathbf{t}_1)_{rm}g_{pumn}(\mathbf{t}_1^T)_{qu}(\mathbf{t}_1^T)_{sn}\\
    &+\sum_{tumn}(\mathbf{t}_1)_{pt}(\mathbf{t}_1)_{rm}g_{tumn}(\mathbf{t}_1^T)_{qu}(\mathbf{t}_1^T)_{sn}
  \end{split}
\end{equation}

This expression can be divided in 16 cases, that can be further simplified by symmetry.
\begin{align}
  {\tilde g}_{ijkl}=&g_{ijkl}+\sum_{e}(g_{iekl}t_j^e+g_{ijke}t_l^e)+\sum_{ef}g_{iekf}t_j^et_l^f\label{eq:t1g_ijkl}\\
%  &=g_{ijkl}+\sum_{e}\bigg[\bigg(g_{iekl}+\sum_fg_{iekf}t_l^f\bigg)t_i^e+g_{ijke}t_j^e\bigg]\\
  {\tilde g}_{ijka}=&g_{ijka}+\sum_{e}g_{ieka}t_j^e\label{eq:t1g_ijka}\\ 
  {\tilde g}_{ijak}=&g_{ijak}+\sum_{e}(g_{ieak}t_j^e+g_{ijae}t_k^e)-\sum_lt_l^ag_{ijlk}\nonumber\\
  &+\sum_{ef}g_{ieaf}t_j^et_l^f-\sum_{el}t_l^a(g_{ielk}t_j^e+g_{ijle}t_k^e)-\sum_{efl}t_l^ag_{ielf}t_j^et_k^f\label{eq:t1g_ijak}\\
  {\tilde g}_{iajk}=&{\tilde g}_{jkia}\label{eq:t1g_iajk}\\
  {\tilde g}_{aijk}=&{\tilde g}_{jkai}\label{eq:t1g_aijk}\\
  {\tilde g}_{ijab}=&g_{ijab}+\sum_{e}g_{ieab}t_j^e-\sum_{l}t_l^ag_{ijlb}-\sum_{el}t_l^ag_{ielb}t_j^e\label{eq:t1g_ijab}\\
  {\tilde g}_{iajb}=&g_{iajb}\label{eq:t1g_iajb}\\
  {\tilde g}_{aijb}=&g_{aijb}+\sum_{e}g_{aejb}t_i^e-\sum_{l}t_l^ag_{lijb}-\sum_{el}t_l^ag_{lejb}t_i^e\label{eq:t1g_aijb}\\
  {\tilde g}_{iabj}=&{\tilde g}_{bjia}\label{eq:t1g_iabj}\\
  {\tilde g}_{aibj}=&g_{aibj}+\sum_{e}(g_{aebj}t_i^e+g_{aibe}t_j^e)-\sum_{l}(t_l^ag_{libj}+t_l^bg_{ailj})\nonumber\\
  &+\sum_{ef}g_{aebf}t_i^et_j^f-\sum_{el}(t_l^ag_{lebj}t_i^e+t_l^bg_{aelj}t_i^e+t_l^ag_{libe}t_j^e+t_l^bg_{aile}t_j^e)\nonumber\\
  &+\sum_{lk}t_l^at_k^bg_{likj}+\sum_{ekl}(t_l^at_k^bg_{lekj}t_i^e+t_l^at_k^bg_{like}t_j^e)\nonumber\\
  &-\sum_{efl}(t_l^ag_{lebf}t_i^et_j^f+t_l^bg_{aelf}t_i^et_j^f)+\sum_{lkef}t_l^at_k^bg_{lekf}t_i^et_j^f\label{eq:t1g_aibj}\\
  {\tilde g}_{abij}=&{\tilde g}_{ijab}\label{eq:t1g_abij}\\
  {\tilde g}_{iabc}=&g_{iabc}-\sum_{l}t_l^bg_{ialc}\label{eq:t1g_iabc}\\
  {\tilde g}_{aibc}=&g_{aibc}+\sum_{e}g_{aebc}t_i^e-\sum_l(t_l^ag_{libc}+t_l^bg_{ailc})\nonumber\\
  &-\sum_{el}(t_l^ag_{lebc}t_i^e+t_l^bg_{aelc}t_i^e)+\sum_{lk}t_l^at_k^bg_{likc}+\sum_{elk}t_l^at_k^bg_{lekc}t_i^e\label{eq:t1g_aibc}\\
  {\tilde g}_{abic}=&{\tilde g}_{icab}\label{eq:t1g_abic}\\
  {\tilde g}_{abci}=&{\tilde g}_{ciab}\label{eq:t1g_abci}\\
  {\tilde g}_{abcd}=&g_{abcd}-\sum_l(t_l^ag_{lbcd}+t_l^cg_{abld})+\sum_{kl}t_l^at_k^cg_{lbkc}\label{eq:t1g_abcd}
\end{align}

Note that, due to symmetry, only 9 cases must be computed ( equations \ref{eq:t1g_ijkl}, \ref{eq:t1g_ijka}, \ref{eq:t1g_ijak}, \ref{eq:t1g_ijab}, \ref{eq:t1g_aijb}, \ref{eq:t1g_aibj}, \ref{eq:t1g_iabc}, \ref{eq:t1g_aibc} and \ref{eq:t1g_abcd}).
Moreover, only part of the equations \ref{eq:t1g_ijkl}, \ref{eq:t1g_aibj} and \ref{eq:t1g_abcd} must be calculated.
If we impose $i\ge k$, $a\ge b$ and $i\ge j$, and $a\ge c$, respectively, we generate all elements for theses three cases.
Last, for the equation \ref{eq:t1g_iajb} there is no transformation and the values can be copied from the original matrix.


\subsection{Residuals}
\hypertarget{sec:ccsd_res}{}
\label{sec:ccsd_res}

\subsubsection{Singles}
\hypertarget{sec:ccsd_res_sing}{}
\label{sec:ccsd_res_sing}


\begin{equation}
  \begin{split}
    \Omega_i^a=\Omega_i^{a(A1)}+\Omega_i^{a(B1)}+\Omega_i^{a(C1)}+\Omega_i^{a(D1)}
  \end{split}
\end{equation}
where
\begin{align}
  \Omega_i^{a(A1)}&=\sum_{ckd}u_{ki}^{cd}{\tilde g}_{adkc}\\
  \Omega_i^{a(B1)}&=-\sum_{ckl}u_{ki}^{ac}{\tilde g}_{kilc}\\
  \Omega_i^{a(C1)}&=\sum_{ck}u_{ik}^{ac}{\tilde F}_{kc}^{I}\\
  \Omega_i^{a(D1)}&={\tilde F}_{ai}^{I}
\end{align}

Other way to open this equation is

\begin{equation}
  \begin{split}
    \Omega_i^a&={\tilde h}_{ai}+\sum_k\{2{\tilde g}_{aikk}-{\tilde g}_{akki}\\
    &+\sum_c[(2t_{ik}^{ac}-t_{ki}^{ac}){\tilde h}_{kc}+(2t_{ik}^{ac}-t_{ki}^{ac})\sum_l(2{\tilde g}_{kcll}-{\tilde g}_{kllc})\\
    &+\sum_d(2t_{ki}^{cd}-t_{ik}^{cd}){\tilde g}_{adkc}-\sum_l(2t_{kl}^{ac}-t_{lk}^{ac}){\tilde g}_{kilc}]\}\\
    &={\tilde h}_{ai}+\sum_{k}\{{\tilde L}_{aikk}+\sum_c[u_{ik}^{ac}{\tilde h}_{kc}+u_{ik}^{ac}\sum_l({\tilde L}_{kcll})+\sum_d(u_{ki}^{cd}{\tilde g}_{adkc})\\
      &-\sum_l(u_{kl}^{ac}{\tilde g}_{kilc})]\}
   \end{split}
\end{equation}

\subsubsection{Doubles}
\hypertarget{sec:ccsd_res_doub}{}
\label{sec:ccsd_res_doub}

\begin{equation}
  \begin{split}
    \Omega_{ij}^{ab}=\Omega_{ij}^{ab(A2)}+\Omega_{ij}^{ab(B2)}+\Omega_{ij}^{ab(C2)}+\Omega_{ji}^{ba(C2)}+\Omega_{ij}^{ab(D2)}+\Omega_{ji}^{ba(D2)}+\Omega_{ij}^{ab(E2)}+\Omega_{ji}^{ba(E2)}
  \end{split}
\end{equation}
where
\begin{align}
  \Omega_{ij}^{ab(A2)}&={\tilde g}_{aibj}+\sum_{cd}t_{ij}^{cd}{\tilde g}_{acbd}\\
  \Omega_{ij}^{ab(B2)}&=\sum_{kl}t_{kl}^{ab}\bigg({\tilde g}_{kilj}+\sum_{cd}{\tilde g}_{kcld}\bigg)\\
  \Omega_{ij}^{ab(C2)}&=-\frac{1}{2}\sum_{ck}t_{kj}^{bc}\bigg({\tilde g}_{kiac}-\frac{1}{2}\sum_{dl}t_{li}^{ad}{\tilde g}_{kdlc}\bigg)-\sum_{ck}t_{ki}^{bc}\bigg({\tilde g}_{kjac}-\frac{1}{2}\sum_{dl}t_{lj}^{ad}{\tilde g}_{kdlc}\bigg)\\
  \Omega_{ij}^{ab(D2)}&=\frac{1}{2}\sum_{ck}u_{jk}^{bc}\bigg({\tilde L}_{aikc}+\frac{1}{2}\sum_{dl}u_{il}^{ad}{\tilde L}_{ldkc}\bigg)\\
  \Omega_{ij}^{ab(E2)}&=\sum_{c}t_{ij}^{ac}\bigg({\tilde F}_{bc}^{I}-\sum_{dkl}u_{ki}^{bd}{\tilde g}_{ldkc}\bigg)-\sum_{k}t_{ik}^{ab}\bigg({\tilde F}_{kj}^{I}+\sum_{cdl}u_{lj}^{cd}{\tilde g}_{kdlc}\bigg)
\end{align}


Other way to open this equation is

\begin{equation}
  \begin{split}
    \Omega_{ij}^{ab}&={\tilde g}_{aibj}\\
    &+\sum_{c}[t_{ij}^{ac}({\tilde h}_{bc}+\sum_{k}2{\tilde g}_{bckk}-{\tilde g}_{bkkc})+t_{ji}^{bc}({\tilde h}_{ac}+\sum_{k}2{\tilde g}_{ackk}-{\tilde g}_{akkc})+\sum_dt_{ij}^{cd}{\tilde g}_{acbd}]\\
    &+\sum_{k}[t_{ik}^{ab}({\tilde h}_{kj}+\sum_{l}2{\tilde g}_{kjll}-{\tilde g}_{kllj})+t_{jk}^{ba}({\tilde h}_{ki}+\sum_{l}2{\tilde g}_{kill}-{\tilde g}_{klli})+\sum_lt_{kl}^{ab}{\tilde g}_{kilj}]\\
    &+\sum_{c}\sum_{k}\{-2^{-1}t_{kj}^{bc}{\tilde g}_{kiac}-t_{ki}^{bc}{\tilde g}_{kjac}-2^{-1}t_{ki}^{ac}{\tilde g}_{kjbc}-t_{kj}^{ac}{\tilde g}_{kibc}\\
    &+(t_{jk}^{bc}-2^{-1}t_{kj}^{bc})(2{\tilde g}_{aikc}-{\tilde g}_{acki})+(t_{ik}^{ac}-2^{-1}t_{ki}^{ac})(2{\tilde g}_{bjkc}-{\tilde g}_{bckj})\\
    &+\sum_{d}\sum_{l}[t_{kl}^{ab}t_{ij}^{cd}+t_{kj}^{bd}t_{li}^{ac}+2^{-1}t_{ki}^{bd}t_{lj}^{ac}+t_{ki}^{ad}t_{lj}^{bc}+2^{-1}t_{kj}^{ad}t_{li}^{bc}\\
    &+2t_{jl}^{bd}t_{ik}^{ac}-t_{jl}^{bc}t_{ik}^{ad}-t_{jl}^{bd}t_{ki}^{ac}-t_{lj}^{bd}t_{ik}^{ac}+2^{-1}t_{jl}^{bc}t_{ki}^{ad}+2^{-1}t_{lj}^{bc}t_{ik}^{ad}+2^{-1}t_{lj}^{bd}t_{ki}^{ac}-4^{-1}t_{lj}^{bc}t_{ki}^{ad}\\
    &+2t_{il}^{ad}t_{jk}^{bc}-t_{il}^{ac}t_{jk}^{bd}-t_{il}^{ad}t_{kj}^{bc}-t_{li}^{ad}t_{jk}^{bc}+2^{-1}t_{il}^{ac}t_{kj}^{bd}+2^{-1}t_{li}^{ac}t_{jk}^{bd}+2^{-1}t_{li}^{ad}t_{kj}^{bc}-4^{-1}t_{li}^{ac}t_{kj}^{bd}\\
    &-t_{ij}^{ad}(2t_{lk}^{bc}-t_{kl}^{bc})-t_{ik}^{ab}(2t_{lj}^{dc}-t_{jl}^{dc}))-t_{ji}^{bd}(2t_{lk}^{ac}-t_{kl}^{ac})-t_{jk}^{ba}(2t_{li}^{dc}-t_{il}^{dc})]{\tilde g}_{kcld}\}
  \end{split}
\end{equation}


\subsection{Amplitudes update}
\hypertarget{sec:ccsd_update}{}
\label{sec:ccsd_update}

\begin{equation}
  t_\mu = t_\mu + \Delta t_\mu
\end{equation}

\begin{equation}
  \Delta t_i^a = \frac{\Omega_i^a(\mathbf{t})}{\epsilon_a - \epsilon_i}
\end{equation}

\begin{equation}
  \Delta t_{ij}^{ab} = \frac{\Omega_{ij}^{ab}(\mathbf{t})}
  {\epsilon_a + \epsilon_b - \epsilon_i - \epsilon_j}
\end{equation}



\newpage
\section{Flowchart}
\label{sec:ccsd_flowchart}

\newcommand{\C}[1]{0.0+#1*4.0}
\newcommand{\T}[1]{0.0-#1*2.5}
\newcommand{\W}[1]{#1*3}

\newcommand{\toBlue}{\color{blue}}
\newcommand{\toBlack}{\color{black}}

\newcommand{\pyVar}[1]{\textit{$\langle$#1$\rangle$}}


\tikzset{
  hyperlink node/.style={
    alias=sourcenode,
    append after command={
      let     \p1 = (sourcenode.north west),
      \p2=(sourcenode.south east),
      \n1={\x2-\x1},
      \n2={\y1-\y2} in
      node [inner sep=0pt, outer sep=0pt,anchor=north west,at=(\p1)] {\hyperlink{#1}{\XeTeXLinkBox{\phantom{\rule{\n1}{\n2}}}}}
      % xelatex needs \XeTeXLinkBox, won't create a link unless it
      % finds text --- rules don't work without \XeTeXLinkBox.
      % Still builds correctly with pdflatex and lualatex
    }
  }
}

\begin{center}
\footnotesize
\begin{tikzpicture}[
  ->,
  double,
  very thick,
  proj/.style={
    align=center,
    font=\bfseries,
    color=red
  },
  projBox/.style={
    thin
  },
  algstep/.style={
    align=center,
    anchor=north,
    rectangle,
    minimum size=6mm,
    rounded corners=3mm,
    very thick,
    draw=black!50,
    top color=white,
    bottom color=black!20
  },
  algmultistep/.style={
    align=center,
    anchor=north,
    rectangle,
    minimum size=6mm,
    rounded corners=3mm,
    very thick,
    draw=black!50,
    top color=white,
    bottom color=white
  }
  ]

  
  \def\lwdB{0.2cm};
  \def\lwd{0.1cm};
  \def\lwdE{0.0cm}


  % ======
  \node (init)  at (\C{0.3}, \T{-1.5})
  [algstep, text width=\W{80}, hyperlink node=sec:]
  {
    \textbf{Initialisation}\\\vspace{\lwdB}
    \begin{minipage}{0.4\textwidth}
      {\footnotesize \texttt{ccsd.}}\\\vspace{\lwd}
      
    \end{minipage}
  };
  % ------

  
  % ======
  \node (ener)  at (\C{-0.0}, \T{0.0})
  [algstep, text width=\W{80}, hyperlink node=sec:ccsd_energy]
  {
    \textbf{Energy}\\\vspace{\lwdB}
    \begin{minipage}{0.4\textwidth}
      {\footnotesize \texttt{ccsd.get\_energy}}\\\vspace{\lwd}
      
    \end{minipage}
  };
  % ------

  
  % ================================================
  \node (res) at (\C{0.0}, \T{1.0})
  [algmultistep, text width=\W{150}, minimum height=4cm, hyperlink node=sec:ccsd_res]
  {
    \textbf{Set residual}\\\vspace{\lwdB}
    \begin{minipage}[t][9cm]{0.4\textwidth}
      \centering
      {\footnotesize \texttt{ccsd.equation}}\\\vspace{\lwd}
    \end{minipage}
  };


  % ======
  \node (resterm1) at (\C{-1.0}, \T{2.0})
  [algstep, text width=\W{50}, hyperlink node=sec:ccsd_res_sing]
  {
    \textbf{Set res. singles ($\Omega_i^a$)}\\\vspace{\lwdB}
    {\footnotesize \texttt{ccsd.equation.res\_singles}}\\\vspace{\lwd}
  };
  % ------

  % ======
  \node (resterm2) at (\C{1.0}, \T{2.0})
  [algstep, text width=\W{50}, hyperlink node=sec:ccsd_res_doub]
  {
    \textbf{Set res. doubles ($\Omega_{ij}^{ab}$)}\\\vspace{\lwdB}
    {\footnotesize \texttt{ccsd.equation.res\_doubles}}\\\vspace{\lwd}
  };
  % ------

  % ======
  \node (resterm3) at (\C{0.0}, \T{3.5})
  [algstep, text width=\W{30}, hyperlink node=sec:residual_eq3]
  {
    \textbf{Set term3}\\\vspace{\lwdB}
    {\footnotesize \texttt{ccsd.equation.term3}}\\\vspace{\lwd}
  };
  % ------

    \draw [->] (resterm1.270) .. controls +(down:1.5cm) and +(up:1.5cm) .. (resterm3.90);
    \draw [->] (resterm2.270) .. controls +(down:1.5cm) and +(up:1.5cm) .. (resterm3.90);
    
  
  % ------------------------------------------------

  
  % ======
  \node (update) at (\C{0.0}, \T{6.0})
  [algstep, text width=\W{90}, hyperlink node=sec:ccsd_update]
  {
    \textbf{Update amplitudes}\\\vspace{\lwdB}
    \begin{minipage}{0.4\textwidth}
      \centering
      {\footnotesize \texttt{ccsd.step}}\\\vspace{\lwd}
    \end{minipage}
  };
  % ------



  \draw [->] (init.270) -- (ener.50);

  \draw [->] (ener.270) .. controls +(down:1.5cm) and +(up:1.5cm) .. (res.90);
  \draw [->] (res.270) .. controls +(down:1.5cm) and +(up:1.5cm) .. (update.90);

  \draw [->] (update.230) -- (\C{-0.3}, \T{7.0})  --  (\C{-2.0}, \T{7.0}) --
  (\C{-2.0}, \T{-0.5}) -- (\C{-0.3}, \T{-0.5}) -- (ener.130);
  % ======
  \node at (\C{-0.5}, \T{6.8}) []
  {$|\mathbf{\Omega}| > \epsilon$};
  % ------

  \draw [->] (update.310) -- (\C{0.3}, \T{7.0})  --  (\C{1.0}, \T{7.0});
  % ======
  \node at (\C{0.5}, \T{6.8}) []
  {$|\mathbf{\Omega}| < \epsilon$};
  % ------

\end{tikzpicture}
\end{center}

 	

%%% Local Variables:
%%% mode: latex
%%% TeX-master: "grassmann_doc.tex"
%%% End:


\newpage
\section{Test: Class Diagram}
Trying to organize the Classes (https://en.wikipedia.org/wiki/Class\_diagram)
%http://www.agilemodeling.com/artifacts/classDiagram.htm#Figure2ConceptualDiagram
\begin{center}
\footnotesize
\begin{tikzpicture}[
  ->,
  double,
  very thick]

\node (test) at (0,0) [class, text width=\W{27}]{\textbf{IntermNormWaveFunction}
\nodepart{second}
+ norm : float\\
+ singles : list[np.array]\\
+ doubles : list[np.array]
\nodepart{third}
+ get\_irrep ( i : int, alpha\_orb : int )\\
+ ij\_from\_N ( N : int )\\
+ N\_from\_ij ( i : int, j : int, i\_irrep : int, j\_irrep : int, exc\_type : str)};

\node (test2) at (\X{40},\Y{0}) [class, text width=\W{27}]{\textbf{WaveFunction}
\nodepart{second}
+ restricted : bool\\
+ point\_group : str\\
+ n\_irre : int\\
+ n\_core : OrbitalsSets\\
+ n\_act : OrbitalsSets\\
+ orb\_dim : OrbitalsSets\\
+ ref\_occ : OrbitalsSets\\
+ n\_ext : OrbitalsSets\\
+ n\_corr\_orb : OrbitalsSets\\
+ n\_alpha, n\_beta, n\_elec, n\_corr\_alpha, n\_corr\_beta, n\_corr\_elec, n\_orb, n\_orb\_nocore: int\\
+ WF\_type : str\\
+ source : str};

\node (test3) at (\X{80},\Y{0}) [class, text width=\W{15}]{\textbf{OrbitalSets}
\nodepart{second}
+ occ\_type : str
};

\draw [-{Triangle[open]}] (test) -- (test2);
\draw [-{Turned Square}] (test3) -- (test2);

\end{tikzpicture}
\end{center}

%%% Local Variables:
%%% mode: latex
%%% TeX-master: "grassmann_doc.tex"
%%% End:



%%% Local Variables:
%%% mode: latex
%%% TeX-master: "grassmann_doc.tex"
%%% End:
