\hypertarget{chap:output}{}
\chapter{Output and log}
\label{sec:output}
\chaptermark{Output and log}
\chapterauthor{}


Grassmann prints main information, such as the final results, in an output file.
Depending on the request and on the calculation,
it may also print extra information in a log file and create extra files in an output directory.
If the output file, the log file, or the output directory already exist,
Grassmann uses the same strategy as Molpro,
renaming these files, appending an ``\verb+_+\emph{X}'' to their names,
where \emph{X} is the lowest positive integer such that nothing is overwritten.

\section{File names}
\label{sec:outname}

The names of the files and directory that Grassmann creates
are determined in the following way:
It can be given with the option \verb+--output+;
Otherwise, if an input file is given, they have the same base name of the input.
Otherwise, if a geometry file is given,
they have the same base name of this geometry file
Otherwise, if a Molpro output file is given,
they have the same base name of this Molpro file.

The output file has the extension \verb+.gr+,
the log file has the extension \verb+.grlog+
and the output directory has the extension \verb+.grdir+, by default.
These extensions can be changed with the options \verb+--out_extension+,
\verb+--log_extension+, and \verb+--dir_extension+.

\section{Controlling log}
\label{sec:controllog}

Logs are printed depending on the log level,
that can be defined by the user with the option \verb+--loglevel+.
The value for this option may be an integer
or a string that represents a numeric value for the log level,
according to the following table.
The interpretation of these strings is case insensitive
and follow the convention of Python's logging module.

\begin{center}
  \begin{tabular}{lc}
    \hline
    level name & numeric value\\
    \hline
    \verb+critical+ & 50 \\
    \verb+error+ & 40\\
    \verb+warning+ & 30\\
    \verb+info+ & 20\\
    \verb+debug+& 10\\
    \hline
  \end{tabular}
\end{center}

Default value is \verb+warning+,
meaning that all logs that have level higher or equal 30 are printed to the log file.
If you decrease the log level, more information is sent to log;
If you increase the log level, less information is sent to log.
For production run, \verb+warning+ is the appropriate choice.
Level \verb+info+ will show some extra information, such as timings,
and might be useful for some initial testing and exploration.
The \verb+debug+ level sends to log extra detailed information on the calculations;
it might be too large.
A log level above warning should be avoided, as it will hide error logs.

In case you want to log what comes from one or few specific function only,
a regular expression can be passed with the option \verb+logfilter+,
and only logs from functions whose name that satisfy this regular expression are tracked.

%%% Local Variables:
%%% mode: latex
%%% TeX-master: "grassmann_doc.tex"
%%% End:


