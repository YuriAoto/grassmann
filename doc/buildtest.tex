\hypertarget{chap:buildtest}{}
\chapter{Building and testing}
\label{sec:buildtest}
\chaptermark{Building and testing}
\chapterauthor{}


\section{Building}

Grassmann is written in Python and Cython.
Although the Python language does not require any compilation,
its extension Cython is actually a wrapper for C code,
with a syntax very similar to Python, and requires compilation.
As long as you have the proper compilers and packages (see Chap.~\ref{sec:requirements}),
the code can be compiled with:
\begin{lstlisting}[style=shstyint]
@\shPr{}@ @\pythonexec{}@ setup.py --build_ext --inplace
\end{lstlisting}

\section{Testing}

The a number of unit, integration, and implementation tests are implemented and it is wise to check if they are working well after building.
To run the essential tests:
\begin{lstlisting}[style=shstyint]
@\shPr{}@ GR_TESTS_CATEG=ESSENTIAL @\pythonexec{}@ setup.py test
\end{lstlisting}

For a finer control of tests (development and debugging) see Chap.~\ref{}.

%%% Local Variables:
%%% mode: latex
%%% TeX-master: "grassmann_doc.tex"
%%% End:


