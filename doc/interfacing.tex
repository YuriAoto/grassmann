\hypertarget{chap:interfacing}{}
\chapter{Interfacing}
\label{sec:interfacing}
\chaptermark{Interfacing}
\chapterauthor{}


\section{Molpro}

Grassmann can read orbitals and some wave functions from \href{https://www.molpro.net/}{Molpro}.
For this to work correctly, some keywords should be added to Molpro's input to trigger the printing
of the desired information.

\subsection{Orbitals}

After the orbital is calculated, it can be exported to an xml file,
with the following line in the Molpro input:
\begin{lstlisting}[style=filestyint]
{put,xml,@\emph{\textrm{name}}@.xml; orbital,@\emph{\textrm{record}}@; nosort; novariables}
\end{lstlisting}
Here, \emph{name} is the base name for the xml file
(recall that Molpro will transform it to lower case)
and \emph{record} is the record where the orbitals are saved.
The part ``\verb+orbital,+\emph{record}\verb+;+'' is optional.
If omitted, the last orbitals will be used.


\subsection{Full CI wave function}

All coefficients of the full CI wave functions must be written to the output.
This could be be achieved after changing the Molpro source code only,
by manually setting the variable \verb+thrres+ to zero, in the file \verb+b/src/fci/fci.F+.


\subsection{CCSD and CISD wave functions}

All amplitudes of the coupled-cluster and configurations interaction wave functions must be written
to the output, what can be achieved with the following line in the beginning of the Molpro input
(just after memory specification):
\begin{lstlisting}[style=filestyint]
gthresh,thrprint=0,printci=0.0
\end{lstlisting}

\section{Integrals from Knizia's IR-WMME}
\label{sec:irwmme}

Molecular integrals are not calculated in Grassmann,
but they are read from the output of the IR-WMME code (see Sect.~\ref{sec:envvar}).
This program is called by Grassmann, and thus it must know where IR-WMME is located.
This is passed to Grassmann through the environment variable \verb+GR_IR_WMME_DIR+.
If the program is correctly compiled and this environment variable is set,
Grassmann should be able to deal with it automatically.

%%% Local Variables:
%%% mode: latex
%%% TeX-master: "grassmann_doc.tex"
%%% End:


