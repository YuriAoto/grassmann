\usepackage[utf8]{inputenc}
\usepackage[english]{babel}

\usepackage[top=2.0cm,left=1.5cm,right=1.0cm,bottom=2.25cm]{geometry}
\linespread{1.25}

%% To add chapter's author. From:
%% https://tex.stackexchange.com/questions/156862/displaying-author-for-each-chapter-in-book
\usepackage{suffix}
\newcommand\chapterauthor[1]{\authortoc{#1}\printchapterauthor{#1}}
\WithSuffix\newcommand\chapterauthor*[1]{\printchapterauthor{#1}}
\makeatletter
\newcommand{\printchapterauthor}[1]{%
  {\parindent0pt\vspace*{-25pt}%
  \linespread{1.1}\large\scshape#1%
  \par\nobreak\vspace*{35pt}}
  \@afterheading%
}
\newcommand{\authortoc}[1]{%
  \addtocontents{toc}{\vskip-10pt}%
  \addtocontents{toc}{%
    \protect\contentsline{chapter}%
    {\hskip1.3em\mdseries\scshape\protect\scriptsize#1}{}{}}
  \addtocontents{toc}{\vskip5pt}%
}
\makeatother


\usepackage[pdftex]{graphicx}
\usepackage{pdflscape}

\usepackage{physics}

\usepackage{multirow}
\usepackage{eufrak}
\usepackage{amsmath}
\usepackage{mathrsfs}
\usepackage{mathtools}
\usepackage{amsfonts}
\usepackage{bm}
\usepackage{bbold}

\usepackage[version=4]{mhchem}
\usepackage[makeroom]{cancel}
\usepackage[table]{xcolor}
\usepackage{capt-of}
\usepackage[colorlinks=true]{hyperref}

\usepackage{tikz}
\usetikzlibrary{calc}
\usetikzlibrary{positioning}
\usetikzlibrary{decorations.text}
\usetikzlibrary{shapes.geometric}
\usetikzlibrary{shapes.multipart}
\usetikzlibrary{arrows.meta}

\newcommand{\toBlue}{\color{blue}}
\newcommand{\toBlack}{\color{black}}

\newcommand{\tikzmark}[1]{\tikz[baseline,remember picture] \coordinate (#1) {};}

\newcommand{\tsc}{\textsuperscript}
\newcommand{\etal}{\textit{et al.}}
\newcommand{\ansatz}{\textit{ansatz}}
\newcommand{\ansatze}{\textit{ans{\"a}tze}}
\newcommand{\mulc}{\multicolumn}
\newcommand{\mulr}{\multirow}
\newcommand{\abinitio}{\textit{ab initio}}
\newcommand{\azero}{a$_0$}
\newcommand{\kcalmol}{kcal mol$^{-1}$}
\newcommand{\degree}{\ensuremath{^\circ}}
\newcommand{\sub}[1]{$_\textrm{#1}$}
\newcommand{\wnumb}{cm$^{-1}$}
\newcommand{\snote}[1]{\marginpar{\scriptsize{\begin{flushleft}#1\end{flushleft}}}}
\newcommand{\JM}{Jeziorski-Monkhorst}
\newcommand{\we}{\ensuremath{\omega_e}}
\newcommand{\wexe}{\ensuremath{\omega_e x_e}}
\newcommand{\integers}{\ensuremath{\mathbb{Z}}}
\newcommand{\rational}{\ensuremath{\mathbb{Q}}}
\newcommand{\realnumb}{\ensuremath{\mathbb{R}}}
\newcommand{\complex}{\ensuremath{\mathbb{C}}}
\newcommand{\projective}{\ensuremath{\mathbb{P}}}
\newcommand{\field}{\ensuremath{\mathbb{K}}}
\newcommand{\x}{\ensuremath{\mathbf{x}}}

\newcommand{\w}{\wedge}

% \newcommand{\bra}[1]{\ensuremath{\left< #1 \right|}}
% \newcommand{\brabig}[1]{\ensuremath{\big< #1 \big|}}
% \newcommand{\braBig}[1]{\ensuremath{\Big< #1 \Big|}}

% \newcommand{\ket}[1]{\ensuremath{\left| #1 \right>}}
% \newcommand{\ketbig}[1]{\ensuremath{\big| #1 \big>}}
% \newcommand{\ketBig}[1]{\ensuremath{\Big| #1 \Big>}}

% \newcommand{\bracket}[2]{\ensuremath{\left< #1 \big| #2 \right>}}
% \newcommand{\bracketbig}[2]{\ensuremath{\big< #1 \big| #2 \big>}}
% \newcommand{\bracketBig}[2]{\ensuremath{\Big< #1 \Big| #2 \Big>}}

% \newcommand{\ketbra}[2]{\ensuremath{\left| #1 \big>\big< #2 \right|}}
% \newcommand{\ketbrabig}[2]{\ensuremath{\big| #1 \big>\big< #2 \big|}}
% \newcommand{\ketbraBig}[2]{\ensuremath{\Big| #1 \Big>\Big< #2 \Big|}}

% \newcommand{\average}[3]{\ensuremath{\left< #1 \big| #2 \big| #3 \right>}}
% \newcommand{\averagebig}[3]{\ensuremath{\big< #1 \big| #2 \big| #3 \big>}}
% \newcommand{\averageBig}[3]{\ensuremath{\Big< #1 \Big| #2 \Big| #3 \Big>}}

\newcommand{\HilbSp}{\ensuremath{\mathscr{H}}}
\newcommand{\Hamilt}{\ensuremath{\mathcal{H}}}
\newcommand{\orbSp}{\ensuremath{\mathcal{V}}}

\newcommand{\up}[1]{\ensuremath{{#1}^\uparrow}}
\newcommand{\down}[1]{\ensuremath{{#1}^\downarrow}}

\newcommand*{\Kmat}{\ensuremath{\mathbf{K}}}
\newcommand*{\Hess}{\ensuremath{\mathbf{H}}}
\newcommand*{\Jac}{\ensuremath{\mathbf{J}}}

\newcommand{\Dss}{\ensuremath{\mathcal{C}}}
\newcommand{\Dmixaa}{\ensuremath{\mathcal{A}}}
\newcommand{\Dmixab}{\ensuremath{\mathcal{B}}}
\newcommand{\Dmix}{\ensuremath{\mathcal{D}}}

\newcommand{\matG}{\ensuremath{{\mathbf{G}}}}
\newcommand{\matM}{\ensuremath{{\bm{\mathcal{M}}}}}
\newcommand{\bigG}{\ensuremath{{\mathscr{G}}}}
\newcommand{\bigH}{\ensuremath{{\mathscr{H}}}}

% Symbols for irreducible representation, irrep
\newcommand{\irp}{\ensuremath{\Gamma}}
\newcommand{\irpP}{{\ensuremath{\Gamma'}}}
\newcommand{\irpPP}{{\ensuremath{\Gamma''}}}
\newcommand{\irpPPP}{{\ensuremath{\Gamma'''}}}
\newcommand{\irpB}{{\ensuremath{\overline{\Gamma}}}}
\newcommand{\irpBB}{{\ensuremath{\overline{\overline{\Gamma}}}}}

\newcommand{\sumijabrestr}{\ensuremath{
    \quad\sum_{\mathclap{\left.\substack{(i>j)\\(a>b)}\right\}\in\irp}}\quad}}
\newcommand{\sumijabfull}{\ensuremath{
    \quad\sum_{\mathclap{\left.\substack{(i,a)\\(j,b)}\right\}\in\irp}}\quad}}
\newcommand{\sumijabmix}{\ensuremath{
    \quad\sum_{\mathclap{\substack{(i,a)\in\irp\\(j,b)\in\irpP}}}\quad}}

\newcommand{\sumB}{\ensuremath{
    \sum_{\irpB \ne \irp}}}
\newcommand{\sumBB}{\ensuremath{
    \sum_{\mathclap{\substack{\irpBB > \irpB\\\irpB \ne \irp\\\irpBB \ne \irp}}}}}

\newcommand{\sumBp}{\ensuremath{
    \sum_{\irpB \ne \irp,\irpP}}}
\newcommand{\sumBBp}{\ensuremath{
    \sum_{\mathclap{\substack{\irpBB > \irpB\\\irpB \ne \irp,\irpP\\\irpBB \ne \irp,\irpP}}}}}

%\newcommand*{\HilbSp}{\ensuremath{\mathscr{H}}}
%\newcommand*{\orbSp}{\ensuremath{\mathcal{W}}}
\newcommand*{\grass}[2]{\ensuremath{\text{Gr}(#1,#2)}}
\newcommand*{\stiefel}[2]{\ensuremath{\text{ST}(#1,#2)}}
\newcommand*{\extAlg}{\ensuremath{\bigwedge \orbSp}}
\newcommand*{\extProd}{\ensuremath{\bigwedge\nolimits^n \orbSp}}
\newcommand*{\extProdTwo}{\ensuremath{\bigwedge\nolimits^2 \orbSp}}


% --------------------------------
% print occupied indices (defined in \orccorb) in \occorbcolor, with \printoccorb
% print virtual indices (defined in \virtcorb) in \virtorbcolor, with \printvirtorb
\newcommand*{\occorb}{ijkl}
\newcommand*{\virtorb}{abcdef}
\newcommand*{\allorb}{pqrsumn}
\newcommand*{\occorbcolor}{blue}
\newcommand*{\virtorbcolor}{red}
\newcommand*{\allorbcolor}{black}
\newcommand*\printoccorb[1] {\textcolor{\occorbcolor} {#1}}
\newcommand*\printvirtorb[1]{\textcolor{\virtorbcolor}{#1}}
\newcommand*\printallorb[1]{\textcolor{\allorbcolor}{#1}}
\makeatletter
\def\instring#1#2{TT\fi\begingroup
  \edef\x{\endgroup\noexpand\in@{#1}{#2}}\x\ifin@}
\newcommand\printorbcolor[1]{
  \if\instring{#1}{\occorb}
  \printoccorb{#1}
  \else 
  \if\instring{#1}{\virtorb}
  \printvirtorb{#1}
  \else
  \printallorb{#1}
  \fi
  \fi
  }
\def\parseorbitals#1#2|{%
    \ifx\relax#1\relax%
    \else%
        \printorbcolor{#1}%
        \edef\@orbitaltmp{#2|}%
            \ifx\relax#2\relax%
            \else%
            \expandafter\parseorbitals\@orbitaltmp%
        \fi%        
    \fi%
}
\newcommand\colororbindices[1]{% Takes in one strings
    % <begin loop>
    % Extracts out one character from both strings
    \edef\@orbitaltmp{#1|}%
    \expandafter\parseorbitals\@orbitaltmp%
    %\printchar{#1}%
    % Continue loop until strings run out of characters
    % <end loop>
}
\makeatother
\newcommand*{\gtil}[1]{\ensuremath{\tilde{g}_{\colororbindices{#1}}}}
\newcommand*{\Ltil}[1]{\ensuremath{\tilde{L}_{\colororbindices{#1}}}}
\newcommand*{\htil}[1]{\ensuremath{\tilde{h}_{\colororbindices{#1}}}}
\newcommand*{\Ftil}[1]{\ensuremath{\tilde{F}^I_{\colororbindices{#1}}}}
\newcommand*{\F}[1]{\ensuremath{F^I_{\colororbindices{#1}}}}
\newcommand*{\h}[1]{\ensuremath{h_{\colororbindices{#1}}}}
\newcommand*{\g}[1]{\ensuremath{g_{\colororbindices{#1}}}}
\newcommand*{\Lmat}[1]{\ensuremath{L_{\colororbindices{#1}}}}
\newcommand*{\orbsum}[1]{\ensuremath{\sum_{\colororbindices{#1}}}}
\newcommand*{\ampl}[2]{\ensuremath{t_{\colororbindices{#1}}^{\colororbindices{#2}}}}
\newcommand*{\uampl}[2]{\ensuremath{u_{\colororbindices{#1}}^{\colororbindices{#2}}}}

% To format our code.
% See https://en.wikibooks.org/wiki/LaTeX/Source_Code_Listings
\usepackage{listings}
\renewcommand{\lstlistingname}{Code}% Listing -> Code
\renewcommand{\lstlistlistingname}{List of \lstlistingname s}
\newcommand{\pythonprompt}{>>>}
\newcommand{\pythonexec}{python3}
\newcommand\numberwithprompt[1]{\footnotesize\ttfamily \pythonprompt}
\newcommand{\shellprompt}{\$}
\newcommand\numberwithpromptSH[1]{\footnotesize\ttfamily \shellprompt}
\usepackage{color}
\definecolor{mygreen}{rgb}{0,0.6,0}
\definecolor{mygray}{rgb}{0.5,0.5,0.5}
\definecolor{mymauve}{rgb}{0.58,0,0.82}
\lstdefinestyle{pysty}{
  language=python,
  basicstyle=\small\ttfamily,
  keywordstyle=\color{blue},
  commentstyle=\color{red},
  stringstyle=\color{mymauve},
  identifierstyle=\color{black},
  numberstyle=\tiny\color{black},
  breaklines=true,
  frame=L,
  keepspaces=true,
  showspaces=false,
  showstringspaces=false,
  showtabs=false,
  tabsize=2,
  extendedchars=true,
  rulecolor=\color{black},
  title=\lstname,
  numbers=left,
  numbersep=10pt,
  stepnumber=1,
  backgroundcolor=\color{white},
}
\lstdefinestyle{pystyint}{
  language=python,
  basicstyle=\small\ttfamily,
  keywordstyle=\color{blue},
  commentstyle=\color{red},
  stringstyle=\color{mymauve},
  identifierstyle=\color{black},
  breaklines=true,
%  captionpos=b,
  frame=none,
  keepspaces=true,
  showspaces=false,
  showstringspaces=false,
  showtabs=false,
  tabsize=2,
  extendedchars=true,
  rulecolor=\color{black},
  title=\lstname,
  backgroundcolor=\color{white},
  numbers=left,
  numberstyle=\numberwithprompt,
}
\newcommand{\shPr}{\$}
\lstdefinestyle{shstyint}{
%  language=bash,
  basicstyle=\small\ttfamily,
  keywordstyle=\color{blue},
  commentstyle=\color{red},
  stringstyle=\color{mymauve},
  identifierstyle=\color{black},
  breaklines=true,
  frame=TB,
  captionpos=tb,
  keepspaces=true,
  showspaces=false,
  showstringspaces=false,
  showtabs=false,
  tabsize=2,
  extendedchars=true,
  rulecolor=\color{black},
  title=\lstname,
  backgroundcolor=\color{white},
  escapechar=@
%%  numbers=left,
%%  numberstyle=\numberwithpromptSH,
}



%%% Local Variables:
%%% mode: latex
%%% TeX-master: "grassmann_doc.tex"
%%% End:
