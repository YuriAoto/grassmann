\documentclass[a4paper,11pt]{article}

\usepackage[top=2.0cm,left=1.5cm,right=1.0cm,bottom=2.25cm]{geometry}
\linespread{1.25}

/home/yuriaoto/Documents/Geom_Elec_Str/notes_on_elec_structure/general_preambl.tex

\begin{document}

\begin{center}
  {\LARGE Optimisation of distance in the Grassmannian to an external wave function}\vspace{1.0cm}

  {\Large Yuri Alexandre Aoto}
\end{center}
{History:
  
  \begin{tabular}{l@{ - }l}
    24 mar 2019 & Start\\
  \end{tabular}
}\vspace{3cm}


\section{Notation}


\begin{center}
\begin{tabular}{ll}
  \hline
  \Hamilt & The electronic Hamiltonian\\
  $n_\alpha$ & The number of alpha electrons\\
  $n_\beta$ & The number of beta electrons\\
  $n$ & The total number of electrons\\
          & $n = n_\alpha + n_\beta$\\
  $\mathcal{V}$ & The space of spatial one-electron wave functions\\
          &(The full space of one-electron wave functions is\\
          &assumed to be a direct sum of two such spaces,\\
          &the first associated to $\alpha$-spin and the second\\
          &to $\beta$-spin electrons. A subscript might be used,\\
          &but the space is the same: $\mathcal{V_\alpha} = \mathcal{V_\beta} = \mathcal{V}$)\\
  $N$ & The dimension of $\mathcal{V}$: $N = \text{dim }\mathcal{V}$\\
  $i$,$j$,$k$,$l$&orbital indices for occupied orbitals\\
          & (the context should make clear the reference determinant)\\
  $a$,$b$,$c$,$d$&orbital indices for virtual orbitals\\
          & (the context should make clear the reference determinant)\\
  $\sigma(p)$ &spin ($\alpha$ or $\beta$) of orbital with index $p$\\
  \HilbSp & The complete Hilbert space of quantum states\\
          &(within finite basis set approximation)\\
          & $\HilbSp = \Lambda^n (\mathcal{V}_\alpha \oplus \mathcal{V}_\beta)$\\
  $\Psi$ & Elements of $\HilbSp$\\
  $\Phi$ & Determinantal elements of $\HilbSp$\\
  $\psi$ & Elements of $\mathcal{V}$\\
  \hline
\end{tabular}
\end{center}

\newpage
\section{Introduction}

Let
\begin{equation}
  \ket{\Psi_\text{FCI}} = \sum_I c_I \ket{\Phi_I}
\end{equation}
be a $n$-electron normalised Full-CI like wave function represented in the orbital basis
\begin{equation}
  \ket{\phi_I} = \phi_{I_1} \wedge \phi_{I_2} \wedge \dots \wedge \phi_{I_n}\,,
\end{equation}
where $\{\phi_p\}$ is an orthonormal basis for the space of one-electron wave functions and $I$ is an ordered multi-index.
We want to find $\ket{\Phi} \in Gr$ such that $|\bracket{\Psi_\text{FCI}}{\Phi}|$ is maximum, where $Gr$ is the image of the Grassmannian in the space of the $n$-electron wave functions.
Recall that
\begin{equation}
  D(\psi_1, \psi_2) = \sqrt{2}\sqrt{1 - |\bracket{\psi_1}{\psi_2}|}
\end{equation}
is a metric in $\projective \HilbSp$.\cite{}
Also, $Gr$ is the set of all elements of $\HilbSp$ that can be written as a single Slater determinant (a decomposable element) for some orthonormal basis of $\mathcal{V}$.

\newpage
\section{Minimisation with respect to FCI}

To find $\ket{\Phi}$, it is equivalent to find the orbitals (namely, a basis of $\mathcal{V}$) such that 
\begin{equation}
  \ket{\Psi_\text{FCI}} = C_0 \ket{\Phi} +
  \sum_{
    \mathclap{\substack{I\\
        I \ne \{1,2,\dots,n\}}}
      }
    c_I \ket{\Phi_I}\,,
\end{equation}
and $|C_0|$ is maximum (over all possible coefficients in all possible basis), since $|C_0| = |\bracket{\Psi_\text{FCI}}{\Phi}|$.

\subsection{Parametrisation by orbital rotations}
We parametrise $Gr$ by the orbital rotations as\cite{}
\begin{equation}
  \ket{\Phi} = e^{-\hat{K}} \ket{\Phi_0}\,,
\end{equation}
where
\begin{eqnarray}
  \hat{K} &=& \sum_{i,a} K_i^a(a_a^\dagger a_i - a_i^\dagger a_a)\\
          &=& \sum_{i,a} K_i^a(a_i^a - a_a^i)\,
\end{eqnarray}
This parametrisation comes from the most general 
\begin{equation}
  \hat{K} = \sum_{p,q} K_q^pa_p^\dagger a_q\,,
\end{equation}
but using that $K_q^p$ is anti-symmetric (so that $e^{-\hat{K}}$ is orthogonal), and excluding rotations within the occupied or virtual spaces of $\ket{\Phi_0}$, that are redundant (do not alter the Slater determinant with $\ket{\Phi_0}$).
For $\hat{K} = 0$ it is clear that $\ket{\Phi} = \ket{\Phi_0}$.

Let
\begin{equation}
  f(K_i^a) = \left|\average{\Psi_\text{FCI}}{e^{-\hat{K}}}{\Phi_0}\right|\,,
\end{equation}
where the argument $K_i^a$ represent all the $n_\alpha(N-n_\alpha) + n_\beta(N-n_\beta)$ elements.
Note that rotations that mix $\alpha$ and $\beta$ orbitals.
We will also assume that $\bracket{\Psi_\text{FCI}}{\Phi_0} > 0$ and this remains true for all steps
of our optimisation.
If $\bracket{\Psi_\text{FCI}}{\Phi_0} < 0$ we of course can change the phase of the wave function and
if $\bracket{\Psi_\text{FCI}}{\Phi_0} = 0$ for the first or any step of the optimisation, we likely started with a very poor initial guess.

\subsection{Jacobian and Hessian}

We want to maximise $f$ and we need its Jacobian and Hessian.
The expressions at $\hat{K} = 0$ are given below.
For the derivation of the expressions, see that hand notes.

\begin{equation}
  \frac{\partial f(\hat{K} = 0)}{\partial K_i^a} = (-1)^{n_{\sigma(i)} - i + 1} C_i^a
\end{equation}

\begin{equation}
  \frac{\partial^2 f(\hat{K} = 0)}{\partial K_i^a \partial K_j^b} =
  \left\{
    \begin{array}{lcr}
      -C_0 & \quad\quad & (i = j, a = b)\\
      0 & \quad\quad & (i \ne j, a = b)\\
      0 & \quad\quad & (i = j, a \ne b)\\
      (-1)^{n_{\sigma(i)}+n_{\sigma(j)}-i-j}C_{ij}^{ab} & \quad\quad & (\sigma(i) \ne \sigma(j))\\
      (-1)^{i+j+1}C_{ij}^{ab} & \quad\quad & (\sigma(i) = \sigma(j), i<j, a<b)\\
      (-1)^{i+j}C_{ij}^{ab} & \quad\quad & (\sigma(i) = \sigma(j), i<j, a>b)\\
    \end{array}
  \right.
\end{equation}
In these equations, $C_i^a$ and $C_{ij}^{ab}$ are the CI coefficients of the single and double excited determinants in the (normalised) wave function $\ket{\Psi_\text{FCI}}$.
The canonical order of the orbitals is assumed to be ``first all $\alpha$, then all $\beta$''.

\subsection{Transformation of the wave function}
In the optimisation process, $\ket{\Phi}$ varies and we would need the Jacobian and the Hessian at $\hat{K} \ne 0$.
The expressions are much more complicated and we avoid this by making a full transformation of $\ket{\Psi_\text{FCI}}$ to the new orbital basis.
Let $U = e^{-\hat{K}}$ be the matrix that transform the orbital basis:
\begin{equation}
  \phi_p = \sum_q \phi'_q U_{qp}\,.
\end{equation}
Given the coefficients $C_I$ of the expansion in the first basis, we want to know the coefficients $C'_I$ such that
\begin{equation}
  \ket{\Psi_\text{FCI}} = \sum_I c_I \ket{\Phi_I} = \sum_I c'_I \ket{\Phi'_I}\,.
\end{equation}
These are given by:
\begin{equation}\label{eq:trans_fci_orbital_basis}
  C'_I = \sum_J C_J\, \text{det}(U_{IJ})\,,
\end{equation}
where $U_{IJ}$ is the minor of the matrix $U$ with the entries in the rows and columns given by the multi-indices $I$ and $J$.

Such transformation is the most time consuming step.


\subsection{Newton-Raphson step}
Starting from a orbital basis $\{\phi_p\}$ such that the first determinant (that is, with the first $n_\alpha$ $\alpha$ orbitals and first $n_\beta$ $\beta$ occupied) is $\ket{\Phi_0}$, we calculate the Jacobian $\mathbf{J}$ and the Hessian $\mathbf{H}$ as shown above.
The Newton step (in the space of the $K_i^a$ parameters) is
\begin{equation}
  \mathbf{z} = -\mathbf{H}^{-1} \mathbf{J}\,.
\end{equation}
From this vector, the operator $\hat{K}$ is constructed and the orbital transformation matrix is given by
\begin{equation}
  U = e^{-\hat{K}}\,.
\end{equation}
This is done for the $\alpha$ and the $\beta$ orbitals and the wave function $\ket{\Psi_\text{FCI}}$ is transformed to the new orbital basis, by equation \ref{eq:trans_fci_orbital_basis}.
This proceeds until convergence.

\end{document}

% \begin{center}
%   \includegraphics[width= \textwidth]{}
%   \captionof{figure}{}
%   \label{fig:}
% \end{center}

  
%%% Local Variables:
%%% ispell-local-dictionary: "british"
%%% mode: latex
%%% TeX-master: t
%%% End:

