\hypertarget{chap:cc_man}{}
\chapter{The coupled-cluster manifold}
\label{sec:cc_man}
\chaptermark{CC manifold}
\chapterauthor{}

{History:
  
  \begin{tabular}{l@{ - }l}
     2021 & Start\\
  \end{tabular}
}\vspace{3cm}


\section{Notation}

\begin{center}
  \begin{tabular}{ll}
    \hline
    \Hamilt    & The electronic Hamiltonian \\
    $\Phi_0$ & The reference Slater determinant \\
    $i,j,k,l,...$ & indices associated to occupied orbitals in $\Phi_0$\\
    $a,b,c,d,...$ & indices associated to virtual orbitals of $\Phi_0$\\
    $\Psi$ & An arbitrary (correlated) wave function\\
    $\Psi_0$ & The FCI wave function associated of the ground state\\
    $\Psi_{CC}$ & An element of the CC-manifold:\\
               & A coupled-cluster like wave function, not necessarily\\
               & that one that solves the CC equations.\\
             & (when the excitation level is implicit by the context\\
                  & or not important)\\
    $\Psi_{CCD}$ & An element of the CCD manifold\\
    $\Psi_{CCSD}$ & An element of the CCSD manifold\\
    \hline
  \end{tabular}
\end{center}

\newpage
\section{Introduction}

In this chapter we describe the main equations used to study how the coupled-cluster manifold
(the CC manifold)
is related to an arbitrary wave function,\footnote{
  These wave functions are not totally arbitrary:
  we will require them to be eigenfunctions of $S_z$.
  Except for this, they will be arbritrary FCI-like wave functions.
  }
usually the exact ground state wave function.
We receive as input this arbitrary wave function ($\Psi_0$) and we compare it with the manifold
of coupled-cluster wave functions ($\Psi_{CC}$).

\section{The distance in the intermediate normalisation}

Let $\Phi_0$ be our reference Slater determinant.
Any state not orthogonal to $\Phi_0$ can be uniquely written in the intermediate normalisation as:
\begin{equation}
  \Psi = \Phi_0 + \sum_{I \ne I_0} c_I \Phi_I\,,
\end{equation}
where $\Phi_I$ are excited Slater determinants (relative to $\Phi_0$).
The distance between two such wave functions is:
\begin{equation}
  \begin{split}
    D_\text{IN}(\Psi,\Psi') &= |\Psi - \Psi'|\\
    &=\sqrt{\sum_{I \ne I_0} (c_I - c'_I)^2 }\\
  \end{split}
\end{equation}

\section{To be in the right side of the CC manifold}

Let $\Psi_0$ be an arbitrary wave function.
We want to check if the CC manifold ``curves towards it'':
We can see a CC  manifold as obtained from the CI manifold (of corresponding type:
CID $\leftrightarrow$ CCD, CISD $\leftrightarrow$ CCSD, etc)
after ``curving'' it.
It is expected that, being CC more accurate that CI, this ``curving'' goes towards the exact
ground state wave function.
This is what we investigate here, and the precise definition is given below.

\section{The CCD manifold}

The CCD manifold is made by all wave functions of the type:
\begin{equation}
  \Psi_{CCD} = e^{T_2} \Phi_0 = (1 + T_2 + \frac{1}{2!}T_2^2 + \frac{1}{3!}T_2^3 + ...) \Phi_0\,,
\end{equation}
where $T_2$ is defined by:
\begin{equation}
  T_2 = \sum_{\mathclap{\substack{
        i<j\\
        a<b}}} t_{ij}^{ab} a_{ij}^{ab}\,.
\end{equation}
Thus, the wave functions at the CCD manifold are linear combinations of the reference ($\Phi_0$) determinant,
of double excitations on top of $\Phi_0$ ($\Phi_{ij}^{ab}$), and higher-rank even excited determinants:
quadruply excited determinants, sextuply excitated determinants, etc.
For each one of these higher-rank excited determinant, its coefficient is given as a sum of products
of $t_{ij}^{ab}$, that are the coefficients of the doubly excited determinants.
Using Szabo's notation, these are
\begin{equation}
  c_{ijkl}^{abcd} = t_{ij}^{ab} * t_{kl}^{cd}\,,
\end{equation}
for quadruply excited determinants,
\begin{equation}
  c_{ijklmn}^{abcdef} = t_{ij}^{ab} * t_{kl}^{cd} * t_{mn}^{ef}\,,
\end{equation}
for sextuply excited determinants, etc.
The way that these are expressed in terms of $t_{ij}^{ab}$ can be found in Lehtola's work \cite{}.

For each one of the higher-rank excited determinant (quadruples, etc),
we say that the CCD manifold bends towards the FCI wave function in that direction if
the sign of $c_I$ in the FCI wave function is the same as of the corresponding cluster expansion
in doubles, using the coefficients of doubles the same wave function.
For instance, if the FCI wave function is given in the intermediate normalization,
the CCD manifold bends toward FCI in the direction of $\Phi_{ijkl}^{abcd}$ if:
\begin{equation}
  \frac{c_{ij}^{ab} * c_{kl}^{cd}}{c_{ijkl}^{abcd}} > 0\,.
\end{equation}

The \emph{vertical distance} between the wave function $\Psi_0$ and the CCD manifold is obtained as
$D_{IN}(\Psi_0, \Psi_0^{CCD,vert})$, where $\Psi_0^{CCD,vert}$ is the CCD wave function
with the same double amplitudes as the coefficients of double excitations as in $\Psi_0$:
\begin{equation}
  D_{CCD}^{vert}(\Psi_0) = 
  D_{IN}(\Psi_0, \Psi_0^{CCD,vert}) = \sqrt{
    \sum_i^a (c_i^a)^2 + \sum_{i<j<k}^{a<b<c} (c_{ijk}^{abc})^2 + \sum_{i<j<k<l}^{a<b<c<d}(c_{ijkl}^{abcd} - c_{ij}^{ab} * c_{kl}^{cd})^2 + \dots
  }
\end{equation}

\section{The CCSD manifold}
We will extend the ideas and equations to the CCSD manifold, formed by the wave functions equation
of the form:
\begin{equation}
  \Psi_{CCD} = e^{T_1 + T_2} \Phi_0 =
  \left(1 + T_1 + T_2 + \frac{1}{2!}(T_1 + T_2)^2 + \frac{1}{3!}(T_1 + T_2)^3 + ...\right) \Phi_0\,,
\end{equation}
where $T_1$ is further defined by:
\begin{equation}
  T_1 = \sum_{i,a} t_i^a a_i^a\,.
\end{equation}

Now the CC wave functions contain not only even rank excitations,
but also singles, triples, etc.
To decompose triple and higher rank excitation in terms of single and doubles can be done
as before, following the work of Lehtola.
However, now we cannot use the coefficients of double and single excitations in $\Psi_0$
directly, since the singles also contribute to the doubles in the CCSD wave function.
Thus,
we have first to extract the contribution of singles from the coefficients of doubles in $\Psi_0$:
\begin{equation}
  \tilde{c}_{ij}^{ab} = c_{ij}^{ab} - c_i^a * c_j^b\,.
\end{equation}

Thus, we say that the CCSD manifold curve to the direction of $\Psi_0$ in the direction of
$\Phi_{ijk}^{abc}$ if:
\begin{equation}
  \frac{c_{ijk}^{abc}}{c_{i}^{a} * \tilde{c}_{jk}^{bc}} > 0\,,
\end{equation}
and that the CCSD manifold curve towards $\Psi_0$ in the direction of
$\Phi_{ijkl}^{abcd}$ if:
\begin{equation}
  \frac{c_{ijkl}^{abcd}}
  {c_{i}^{a} * c_{j}^{b} * c_{k}^{c} * c_{l}^{d}
    + c_{i}^{a} * c_{j}^{b} * \tilde{c}_{kl}^{cd}
    + \tilde{c}_{ij}^{ab} * \tilde{c}_{kl}^{cd}} > 0\,.
\end{equation}

Finally the \emph{vertical distance} between the wave function $\Psi_0$ and the CCSD manifold is obtained as
$D_{IN}(\Psi_0, \Psi_0^{CCSD,vert})$, where $\Psi_0^{CCSD,vert}$ is the CCSD wave function
with the same single amplitudes as the coefficients of single excitations as in $\Psi_0$,
but doubles amplitudes given by $\tilde{c}_{ij}^{ab}$:
\begin{equation}
  \begin{split}
    D_{CCSD}^{vert}(\Psi_0) &= 
    D_{IN}(\Psi_0, \Psi_0^{CCSD,vert})\\
    &= \sqrt{
      \sum_{i<j<k}^{a<b<c}
      (c_{ijk}^{abc} - c_{i}^{a} * \tilde{c}_{jk}^{bc})^2
      + \sum_{i<j<k<l}^{a<b<c<d}
      (c_{ijkl}^{abcd}
      - c_{i}^{a} * c_{j}^{b} * c_{k}^{c} * c_{l}^{d}
      - c_{i}^{a} * c_{j}^{b} * \tilde{c}_{kl}^{cd}
      - \tilde{c}_{ij}^{ab} * \tilde{c}_{kl}^{cd})^2 + \dots
    }
  \end{split}
\end{equation}




%%% Local Variables:
%%% mode: latex
%%% TeX-master: "grassmann_doc.tex"
%%% End:
