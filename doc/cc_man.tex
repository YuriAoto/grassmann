\hypertarget{chap:cc_man}{}
\chapter{The coupled-cluster manifold}
\label{sec:cc_man}
\chaptermark{CC manifold}
\chapterauthor{}

{History:
  
  \begin{tabular}{l@{ - }l}
     2021 & Start\\
  \end{tabular}
}\vspace{3cm}


\section{Notation}

\begin{center}
  \begin{tabular}{ll}
    \hline
    \Hamilt    & The electronic Hamiltonian \\
    $\Phi_0$ & The reference Slater determinant \\
    $i,j,k,l,...$ & indices associated to occupied orbitals in $\Phi_0$\\
    $a,b,c,d,...$ & indices associated to virtual orbitals of $\Phi_0$\\
    $\Psi$ & An arbitrary (correlated) wave function\\
    $\Psi_0$ & The FCI wave function associated of the ground state\\
    $\Psi_{CC}$ & An element of the CC-manifold:\\
               & A coupled-cluster like wave function, not necessarily\\
               & that one that solves the CC equations.\\
             & (when the excitation level is implicit by the context\\
                  & or not important)\\
    $\Psi_{CCD}$ & An element of the CCD manifold\\
    $\Psi_{CCSD}$ & An element of the CCSD manifold\\
    \hline
  \end{tabular}
\end{center}

\newpage
\section{Introduction}

In this chapter we describe the main equations used to study how the coupled-cluster manifold
(the CC manifold)
is related to an arbitrary wave function,\footnote{
  These wave functions are not totally arbitrary:
  we will require them to be eigenfunctions of $S_z$.
  Except for this, they will be arbritrary FCI-like wave functions.
  }
usually the exact ground state wave function.
We receive as input this arbitrary wave function ($\Psi_0$) and we compare it with the manifold
of coupled-cluster wave functions ($\Psi_{CC}$).

\section{The distance in the intermediate normalisation}

Let $\Phi_0$ be our reference Slater determinant.
Any state not orthogonal to $\Phi_0$ can be uniquely written in the intermediate normalisation as:
\begin{equation}
  \Psi = \Phi_0 + \sum_{I \ne I_0} c_I \Phi_I\,,
\end{equation}
where $\Phi_I$ are excited Slater determinants (relative to $\Phi_0$).
The distance between two such wave functions is:
\begin{equation}
  \begin{split}
    D_\text{IN}(\Psi,\Psi') &= |\Psi - \Psi'|\\
    &=\sqrt{\sum_{I \ne I_0} (c_I - c'_I)^2 }\\
  \end{split}
\end{equation}

\section{To be in the right side of the CC manifold}

Let $\Psi_0$ be an arbitrary wave function.
We want to check if the CC manifold ``curves towards it'':
We can see a CC  manifold as obtained from the CI manifold (of corresponding type:
CID $\leftrightarrow$ CCD, CISD $\leftrightarrow$ CCSD, etc)
after ``curving'' it.
It is expected that, being CC more accurate that CI, this ``curving'' goes towards the exact
ground state wave function.
This is what we want to investigate, and the precise definition is given below.

\section{The vertical distance and the vertical ``projection''}

\subsection{The CCD manifold}

The CCD manifold is made by all wave functions of the type:
\begin{equation}
  \Psi_{CCD} = e^{T_2} \Phi_0 = (1 + T_2 + \frac{1}{2!}T_2^2 + \frac{1}{3!}T_2^3 + ...) \Phi_0\,,
\end{equation}
where $T_2$ is defined by:
\begin{equation}
  T_2 = \sum_{\mathclap{\substack{
        i<j\\
        a<b}}} t_{ij}^{ab} a_{ij}^{ab}\,.
\end{equation}
Thus, the wave functions at the CCD manifold are linear combinations of the reference ($\Phi_0$) determinant,
of double excitations on top of $\Phi_0$ ($\Phi_{ij}^{ab}$), and higher-rank even excited determinants:
quadruply excited determinants, sextuply excitated determinants, etc.
For each one of these higher-rank excited determinant, its coefficient is given as a sum of products
of $t_{ij}^{ab}$, that are the coefficients of the doubly excited determinants.
Using Szabo's notation, these are
\begin{equation}
  c_{ijkl}^{abcd} = t_{ij}^{ab} * t_{kl}^{cd}\,,
\end{equation}
for quadruply excited determinants,
\begin{equation}
  c_{ijklmn}^{abcdef} = t_{ij}^{ab} * t_{kl}^{cd} * t_{mn}^{ef}\,,
\end{equation}
for sextuply excited determinants, etc.

The way that these are expressed in terms of $t_{ij}^{ab}$ can be obtained by expanding the operator $e^{T_2}$.
This has been done by Lehtola and coauthors \cite{}.
They report in their paper the \emph{cluster decomposition} of the FCI wave function,
and show how each CI coefficient is decomposed in amplitudes.
for the case with no truncation in the cluster operator $T$).
Whenever we consider only the $T_2$ operator,
we can simply neglect all contribution from amplitudes other than doubles.
For instance:
\begin{equation}
  \begin{split}
    c_{ijkl}^{abcd} =
    & \phantom{+} t_{ij}^{ab}t_{kl}^{cd} - t_{ij}^{ac}t_{kl}^{bd} + t_{ij}^{ad}t_{kl}^{bc}\\
  & + t_{ij}^{bc}t_{kl}^{ad} - t_{ij}^{bd}t_{kl}^{ac} + t_{ij}^{cd}t_{kl}^{ab}\\
  & - t_{ik}^{ab}t_{jl}^{cd} + t_{ik}^{ac}t_{jl}^{bd} - t_{ik}^{ad}t_{jl}^{bc}\\
  & - t_{ik}^{bc}t_{jl}^{ad} + t_{ik}^{bd}t_{jl}^{ac} - t_{ik}^{cd}t_{jl}^{ab}\\
  & + t_{il}^{ab}t_{jk}^{cd} - t_{il}^{ac}t_{jk}^{bd} + t_{il}^{ad}t_{jk}^{bc}\\
  & + t_{il}^{bc}t_{jk}^{ad} - t_{il}^{bd}t_{jk}^{ac} + t_{il}^{cd}t_{jk}^{ab}
  \end{split}
\end{equation}
This rules for the decomposition until eight-rank excitations can be obtained
from Lehtola's Github repository.
The decomposition is done in \texttt{wave\_functions.fci.cluster\_decompose}.
Note, however, that the above rules hold when we consider occupied orbitals
in the reference coming before the virtuals (for each alpha or beta parts).
If the orbitals are ordered in a different way a sign correction should be applied.

For each one of the higher-rank excited determinant (quadruples, etc),
we say that the CCD manifold bends towards the FCI wave function in that direction if
the sign of $c_I$ in the FCI wave function is the same as of the corresponding cluster expansion
in doubles, using the coefficients of doubles of the same wave function.
For instance, if the FCI wave function is given in the intermediate normalization,
the CCD manifold bends toward FCI in the direction of $\Phi_{ijkl}^{abcd}$ if:
\begin{equation}
  \frac{c_{ij}^{ab} * c_{kl}^{cd}}{c_{ijkl}^{abcd}} > 0\,.
\end{equation}
Furthermore,
the \emph{vertical distance} between the wave function $\Psi_0$ and the CCD manifold is obtained as
\begin{equation}
  D_\text{IN}(\Psi_0, \Psi_0^{CCD,vert})\,,
\end{equation}
where $\Psi_0^{CCD,vert}$ is the CCD wave function
with the same double amplitudes as the coefficients of double excitations as in $\Psi_0$:
\begin{equation}
  \begin{split}
    D_{CCD}^{vert}(\Psi_0) =& D_{IN}(\Psi_0, \Psi_0^{CCD,vert})\\
    =& \sqrt{
      \sum_i^a (c_i^a)^2
      + \sum_{i<j<k}^{a<b<c} (c_{ijk}^{abc})^2
      + \sum_{i<j<k<l}^{a<b<c<d}(c_{ijkl}^{abcd} - c_{ij}^{ab} * c_{kl}^{cd})^2 + \dots
    }
  \end{split}
\end{equation}

The calculation of the vertical distance between $\Psi_0$ and the CCD manifold,
as well as the check if the manifold curves towards $\Psi_0$,
is made at \texttt{fci.WaveFunctionFCI.compare\_to\_CC\_manifold}.

\subsection{The CCSD manifold}
We will extend the ideas and equations discussed above to the CCSD manifold,
formed by the wave functions of the form:
\begin{equation}
  \Psi_{CCSD} = e^{T_1 + T_2} \Phi_0 =
  \left(1 + T_1 + T_2 + \frac{1}{2!}(T_1 + T_2)^2 + \frac{1}{3!}(T_1 + T_2)^3 + ...\right) \Phi_0\,,
\end{equation}
where $T_1$ is further defined by:
\begin{equation}
  T_1 = \sum_{i,a} t_i^a a_i^a\,.
\end{equation}

Now the CC wave functions contain not only even rank excitations,
but also singles, triples, etc.
The decomposition triple and higher rank excitations in terms of single and doubles \emph{amplitudes}
can be done as before, following the work of Lehtola.
However, now we cannot use the coefficients of double and single excitations in $\Psi_0$
directly, since the amplitudes of singles also contribute to the coefficient of doubly excited
determinants in the CCSD wave function
Thus, we have first to extract the contribution of singles
from the coefficients of doubles in $\Psi_0$:
\begin{equation}\label{eq:extract_S_from_D}
  t_{ij}^{ab} = c_{ij}^{ab} - c_i^a * c_j^b\,.
\end{equation}
Note that $c_i^a = t_i^a$.

Thus, we say that the CCSD manifold curves towards $\Psi_0$ in the direction of
$\Phi_{ijk}^{abc}$ if
\begin{equation}
  \frac{c_{ijk}^{abc}}{c_{i}^{a} * t_{jk}^{bc}} > 0\,,
\end{equation}
and that the CCSD manifold curves towards $\Psi_0$ in the direction of
$\Phi_{ijkl}^{abcd}$ if
\begin{equation}
  \frac{c_{ijkl}^{abcd}}
  {c_{i}^{a} * c_{j}^{b} * c_{k}^{c} * c_{l}^{d}
    + c_{i}^{a} * c_{j}^{b} * t_{kl}^{cd}
    + t_{ij}^{ab} * t_{kl}^{cd}} > 0\,.
\end{equation}

Finally the \emph{vertical distance} between the wave function $\Psi_0$
and the CCSD manifold is obtained as
$D_\text{IN}(\Psi_0, \Psi_0^{CCSD,vert})$, where $\Psi_0^{CCSD,vert}$ is the CCSD wave function
with the same single amplitudes as the coefficients of single excitations as in $\Psi_0$,
but doubles amplitudes given by Equation~\eqref{eq:extract_S_from_D}:
\begin{equation}
  \begin{split}
    D_{CCSD}^{vert}(\Psi_0) &= 
    D_{IN}(\Psi_0, \Psi_0^{CCSD,vert})\\
    &= \sqrt{
      \sum_{i<j<k}^{a<b<c}
      (c_{ijk}^{abc} - c_{i}^{a} * t_{jk}^{bc})^2
      + \sum_{i<j<k<l}^{a<b<c<d}
      (c_{ijkl}^{abcd}
      - c_{i}^{a} * c_{j}^{b} * c_{k}^{c} * c_{l}^{d}
      - c_{i}^{a} * c_{j}^{b} * t_{kl}^{cd}
      - t_{ij}^{ab} * t_{kl}^{cd})^2 + \dots
    }
  \end{split}
\end{equation}

\section{The distance to the coupled-cluster manifold}

For the CI manifold, the vertical projection of an arbitrary wave function
is the point of the manifold that maximises the distance between the manifold
and the wave function, that is just the vertical distance.
For the coupled-cluster manifold, however, this is no longer true, since it is curved.
We will assume that there is a point in the coupled cluster manifold that minimises the
distance (in the intermediate normalisation) to a given wave function.
Since this manifold is not compact this is not necessarily true,
but seems a reasonable assumption if the wave function is the exact ground state
wave function for a well behaved system.
To find the point at this manifold that minimises this distance a optimisation has to be done.

The goal is to find $\Psi_{CC}^{minD}$ at the CC manifold (of a given level)
such that
\begin{equation}
  D_{IN}(\Psi_0, \Psi_{CC})^2 = \braket{\Psi_0 - \Psi_{CC}}{\Psi_0 - \Psi_{CC}}
\end{equation}
is minimum for all $\Psi_{CC} \in U_{CC}$.
Minimising the square of the distance is just a matter of convenience.
Since the manifold is already parametrised by the amplitudes,
we will use them, stored in a vector $\mathbf{t}$ whose elements are $t_\rho$.
Applying the stationary condition we have:
\begin{equation}
  \begin{split}
    0 =& \frac{1}{2} \frac{\partial}{\partial t_\rho} D_{IN}(\Psi_0, \Psi_{CC}(\mathbf{t}))^2\\
    =& \mel{\Psi_0 - \Psi_{CC}(\mathbf{t})}{e^{T(\mathbf{t})}}{\Phi_\rho}\\
    =& \mel{\Psi_0 - \Psi_{CC}(\mathbf{t})}{\tau_\rho}{\Psi_{CC}(\mathbf{t})} = J_\rho(\mathbf{t})\,.
  \end{split}
\end{equation}
The factor $\frac{1}{2}$ is just for convenience again.
The wave function $\Psi_0$ must be in the intermediate normalisation.
$\Phi_\rho = \tau_\rho\Phi_0$ and the last equality holds because $\tau_\rho$ commutes with
the cluster operator.
The Hessian, needed for a Newton optimisation:
\begin{equation}
  \begin{split}
    H_{\sigma\rho}(\mathbf{t}) =
    & \mel{\Psi_0 - \Psi_{CC}(\mathbf{t})}{e^{T(\mathbf{t})}}{\Phi_{\rho\sigma}}
    - \mel{\Phi_\sigma}{e^{T(\mathbf{t})^\dagger}e^{T(\mathbf{t})}}{\Phi_\rho}\\
    & \mel{\Psi_0 - \Psi_{CC}(\mathbf{t})}{\tau_\sigma\tau_\rho}{\Psi_{CC}(\mathbf{t})}
    - \mel{\Psi_{CC}(\mathbf{t})}{\tau_\sigma^\dagger\tau_\rho}{\Psi_{CC}(\mathbf{t})}\,.
  \end{split}
\end{equation}

If these quantities are calculated, the Newton step is obtained as:
\begin{equation}
  \mathbf{t}_\text{new} =
  \mathbf{t} - \left(\mathbf{H}(\mathbf{t})\right)^{-1}\mathbf{J}(\mathbf{t})\,.
\end{equation}
This is implemented at \texttt{wave\_functions.fci.WaveFunctionFCI.calc\_dist\_to\_CC\_manifold}.
What is more challenging, and time consuming, is the calculation of $\mathbf{J}(\mathbf{t})$ and
$\mathbf{H}(\mathbf{t})$.


%%% Local Variables:
%%% mode: latex
%%% TeX-master: "grassmann_doc.tex"
%%% End:
